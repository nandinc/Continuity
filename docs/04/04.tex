\documentclass[12pt,a4paper,oneside]{article}
\usepackage[utf8]{inputenc}
\usepackage{t1enc} % hyphenate accented chars
\usepackage[hungarian]{babel}
\usepackage{../fedlap}
\usepackage{fancyhdr} % elofej, elolab
\usepackage{graphicx}
\usepackage{datetime} % specify date format
\setcounter{secnumdepth}{3} % enable subsubsection

% hasonlitson a doc verziora
\addtolength{\voffset}{-1cm}

% cim
\csapat{nand}{39}
\konzulens{Bozóki Szilárd}
\datum{\todaynum}

% csapattagok
\taga{Berki Endre}{HQNHER}{berkiendre@gmail.com}
\tagb{Fodor Bertalan Ferenc}{H4T1UX}{foberci@gmail.com}
\tagc{Kádár András}{JFENWR}{arycika@gmail.com}
\tagd{Thaler Benedek}{EDDO10}{thalerbenedek@gmail.com}

\setlength{\headheight}{1.3em}
\setlength{\headsep}{2em}

% elofej, elolab
\fancyhf{}

\fancyhead[OL] { \tiny \leftmark{} }
\fancyhead[OR] { \tmpcsapat }

\fancyfoot[OR] { \thepage }
\fancyfoot[OL] { \tmpdatum }

\pagestyle{fancy}

% custom date format, according to customer request
% you have to use the \todaynum command instead of today,
% becouse babel overrides it, and I couldn't find a way to override
% it again. I was tempted to call this format \todaybozoki
\newcommand{\todaynum}{\the\year. \twodigit\month. \twodigit\day}


\begin{document}

\anyag{4. Analízis modell kidolgozása}
\fedlap

\addtocounter{section}{3}
\section{Analízis modell kidolgozása}

	\subsection{Objektum katalógus}
%GENERATOR:CLASS_RESPONSIBILITY
		\begin{description}
			\item[AbstractFrameItem] Alapértelmezett megvalósítása egy keretben lévő objektumnak.
			\item[Area] Területrészt leíró osztály. Összehasonlítható egy másik osztállyal, hogy fedik-e egymást.
			\item[DIRECTION] Irányokat jelöl a két dimenziós térben.
			\item[Door] Ajtó objektum, melyet ha megérint a Stickman arról esemény bocsát ki.
			\item[Frame] A pálya által alkotott táblázat egy cellája, amely elemeket tartalmaz. Ez felelős az elemek (Stickman) mozgatásáért kereten belül és között egyaránt. Két elem egy helyen való tartózkodásáról értesíti az elemeket (collision notify).
			\item[FrameItem] A keretben elhelyezkedő elemek által megvalósított iterfész, melyen keresztül a keret menedzselni tudja a benne lévő elemeket.
			\item[Game] A játékot reprezentáló objektum, amely kezeli az aktuális pályát.
			\item[Key] Kulcs elem, melyet megérintve a Stickman meg tud szerezni. Erről egy esemény küldésén keresztül értesíti a külvilágot.
			\item[Map] Számon tartja a pályán elhelyezett kulcsokat számát, valamint a már összegyűjtött kulcsok számát. Felelős a keretek mozgatásáért, amit kommunikácó nélkül meg tud valósítani.
			\item[MapFactory] Felelős a pályák létrehozásáért, bennük a keretek és az elemek elhelyezéséért.
			\item[Platform] Olyan elem, mellyel nem tud a Stickman egy helyen tartózkodni, azaz korlátozza a Stickman mozgásterét.
			\item[PubSub] Üzenetközvetítő osztály, mely feliratkozásokat tart számon, és ha valakitől eseményt kap, arról értesíti az arra feliratkozottakat.
			\item[Stickman] A játékos által irányított figura, mely a pályán mozog.
			\item[Subscriber] Olyan interfész, melyen keresztül eseményeket lehet fogadni a PubSubtól.
			\item[Timer] Időzítésért felelős osztály, bizonyos időközönként kibocsát egy 'tick' eseményt az átadott PubSub objektumra.
		\end{description}
%GENERATOR:CLASS_RESPONSIBILITY	
	
	\subsection{Osztályok leírása}
	
%GENERATOR:CLASS_DESCRIPTIONS
		\subsubsection{AbstractFrameItem} Absztrakt osztály.
				 Alapértelmezett megvalósítása egy keretbe helyezett objektumnak.   Nyilvántartja a kereten belüli pozícióját, valamint hordoz  egy referenciát a tartalmazó keretre. 			\begin{description}


				\item[Ősosztályok] (nincs).
				\item[Interfészek] FrameItem.
				\item[Attribútumok]$\ $
					\begin{description}
						\item[\texttt{protected Area area}] A tartalmazó kereten belül elfoglalt pozíció 
						\item[\texttt{protected Frame frame}] A tartalmazó keret 
					\end{description}
				\item[Metódusok] (nincs)
			\end{description}

		\subsubsection{Area}
				 Egy tetszőleges keret egy területét leíró osztály.  Ez az osztály elfedi, hogy egy keret valójában hány dimenziós,  valamint azt, hogy egy keret egy összefüggő ponthalmaza milyen   paraméterekkel írható le. 			\begin{description}


				\item[Ősosztályok] (nincs).
				\item[Interfészek] (nincs)
				\item[Attribútumok]$\ $
					\begin{description}
						\item[\texttt{private int height}] A terület magassága 
						\item[\texttt{private int width}] A terület szélessége 
						\item[\texttt{private int x}] A terület bal felső sarkának x eltolása 
						\item[\texttt{private int y}] A terület bal felső sarkának y eltolása 
					\end{description}
				\item[Metódusok]$\ $
					\begin{description}
%						\item[\texttt{public int getHeight()}] \hfill \\
						% TODO document getHeight
%						\item[\texttt{public int getWidth()}] \hfill \\
						% TODO document getWidth
%						\item[\texttt{public int getX()}] \hfill \\
						% TODO document getX
%						\item[\texttt{public int getY()}] \hfill \\
						% TODO document getY
						\item[\texttt{public boolean hasCollision(Area area)}] \hfill \\ Ellenőrzi, hogy a kapott Area objektummal van-e közös pontja.  A metódus feltételezi, hogy a két terület azonos keretben található. 
%						\item[\texttt{public void setHeight(int height)}] \hfill \\
						% TODO document setHeight
%						\item[\texttt{public void setWidth(int width)}] \hfill \\
						% TODO document setWidth
%						\item[\texttt{public void setX(int x)}] \hfill \\
						% TODO document setX
%						\item[\texttt{public void setY(int y)}] \hfill \\
						% TODO document setY
					\end{description}
			\end{description}

		\subsubsection{DIRECTION}
				 Enumeráció. Elfedi, hogy a játéktér hány dimenziós.  A tartalmazott elemek leírják az összes lehetséges irányt,   melybe a játékos mozoghat, vagy keretet cserélhet.  			\begin{description}


				\item[Ősosztályok] (nincs).
				\item[Interfészek] (nincs)
				\item[Attribútumok] (nincs)
				\item[Metódusok] (nincs)
			\end{description}

		\subsubsection{Door}
				 A pályák befejezésére szolgáló ajtót reprezentálja.  Ha a Stickman az összes kulcs birtokában megérinti,  a pálya teljesítésre kerül. 			\begin{description}


				\item[Ősosztályok] AbstractFrameItem $\rightarrow{}$ Door.
				\item[Interfészek] (nincs)
				\item[Attribútumok] (nincs)
				\item[Metódusok] (nincs)
			\end{description}

		\subsubsection{Frame}
				 Egy pályán belüli keret, amelyben mozoghat a Stickman. Tartalmazza a pálya többi elemét (Door, Key, Platform). 			\begin{description}


				\item[Ősosztályok] (nincs).
				\item[Interfészek] (nincs)
				\item[Attribútumok] (nincs)
				\item[Metódusok]$\ $
					\begin{description}
						\item[\texttt{public void addItem(FrameItem item)}] \hfill \\ Hozzáadja a megadott elemet a kerethez. 
						\item[\texttt{protected boolean checkCollision(Area area)}] \hfill \\ Ellenőrzi, hogy a megadott területen  található-e szilárd objektum. 
						\item[\texttt{protected boolean isTraversable(Frame frame, DIRECTION d)}] \hfill \\ Megállapítja, hogy a megkapott Keret és saját  maga között fennáll-e az átjárhatóság a  megadott irányban. 
						\item[\texttt{public void removeItem(FrameItem item)}] \hfill \\ Eltávolítja a megadott elemet a keretből 
						\item[\texttt{public boolean requestArea(FrameItem item, Area area)}] \hfill \\ A metódust hívó elem kérést intéz a kerethez,  hogy el szeretné foglalni a megadott területet.  A keret felelőssége a terület ellenőrzése, és szabad  terület esetén az elem pozíciójának frissítése. 
					\end{description}
			\end{description}

		\subsubsection{FrameItem} Interfész.
				 A Frame osztály ezen a felületen keresztül éri el a tartalmazott objektumokat.   Ezen interfacet valósítják meg a pályán lévő objektumok (Platform, Door, Key, Stickman) 			\begin{description}


				\item[Ősosztályok] (nincs).
				\item[Metódusok]$\ $
					\begin{description}
						\item[\texttt{public void collision(FrameItem colliding)}] \hfill \\ A tartalmazó keret jelezheti ezen a metóduson keresztül,  hogy egy másik elem, melyet paraméterül ad,  hozzáért (collision) ehez az elemhez. 
						\item[\texttt{public Area getArea()}] \hfill \\ Visszaadja az elem pozícióját a kereten belül. 
						\item[\texttt{public boolean isSolid()}] \hfill \\ Megadja, hogy az elem szilárd-e vagy sem.  Ez a kereten belüli mozgások esetén az  ütközések ellenőrzésekor használatos. 
						\item[\texttt{public void setArea(Area area)}] \hfill \\ Beállítja az elem pozícióját a kereten belül. 
						\item[\texttt{public void setFrame(Frame frame)}] \hfill \\ Beállítja az elemet tartalmazó keretet. 
					\end{description}
			\end{description}

		\subsubsection{Game}
				 A játékot szervező objektum. Felelős az új pályák betöltéséért és a teljesített pályák törléséért. 			\begin{description}


				\item[Ősosztályok] (nincs).
				\item[Interfészek] (nincs)
				\item[Attribútumok]$\ $
					\begin{description}
						\item[\texttt{protected Map currentMap}] Az aktuális pálya 
					\end{description}
				\item[Metódusok]$\ $
					\begin{description}
						\item[\texttt{public void loadMap(int mapId)}] \hfill \\ Betölti a megadott pályát. 
					\end{description}
			\end{description}

		\subsubsection{Key}
				 A pályán található kulcsokat reprezentáló objektum. 			\begin{description}


				\item[Ősosztályok] AbstractFrameItem $\rightarrow{}$ Key.
				\item[Interfészek] (nincs)
				\item[Attribútumok]$\ $
					\begin{description}
						\item[\texttt{private boolean collected}] A kulcs összegyűjtöttségének állapotát tárolja. 
					\end{description}
				\item[Metódusok]$\ $
					\begin{description}
						\item[\texttt{public boolean isCollected()}] \hfill \\ Megadja, hogy megszerezték-e a kulcsot. 
					\end{description}
			\end{description}

		\subsubsection{Map}
				 A pályákat reprezentálja. Tartalmazza a kereteket és azok elhelyezkedését,  kontrolállja az átrendezésüket, melyet kommunikáció nélkül meg tud tenni, mivel a keretek nem tudnak relatív elhelyezkedésükről. 			\begin{description}


				\item[Ősosztályok] (nincs).
				\item[Interfészek] (nincs)
				\item[Attribútumok]$\ $
					\begin{description}
						\item[\texttt{protected Collection<Frame> frames}] A pályához tartozó kereteket tárolja. A Map osztály  meg tudja állapítani a keretek közötti szomszédossági  viszonyokat a gyűjtemény alapján. 
					\end{description}
				\item[Metódusok]$\ $
					\begin{description}
						\item[\texttt{public void addItem(FrameItem item)}] \hfill \\ Hozzáadja a megadott elemet az elem által specifikált pozícióhoz.  Amennyiben a pálya inicializálása során az adott helyen még nincs keret,  létrehoz egyet.   A hozzáadott elem pozícióját megváltoztatja úgy, hogy az relatív  legyen a tartalmazó kerethez. 
						\item[\texttt{public Frame getNeighbour(Frame caller, DIRECTION direction)}] \hfill \\ Visszaadja a megadott keret direction irányba található  szomszédját. null-t ad vissza, ha a megadott irányban  nincs szomszéd. 
						\item[\texttt{public void moveFrame(DIRECTION d)}] \hfill \\ Kicseréli az üres helyet a megadott iránnyal ellentétes  szomszédjával. 
					\end{description}
			\end{description}

		\subsubsection{MapFactory}
				 Azonosító alapján Pályákat szolgáltat.  Létrehozza a pályát, majd az azonosító alapján feltölti  a megfelelő objektumokkal. 			\begin{description}


				\item[Ősosztályok] (nincs).
				\item[Interfészek] (nincs)
				\item[Attribútumok] (nincs)
				\item[Metódusok]$\ $
					\begin{description}
						\item[\texttt{public Map getMap(int mapId, PubSub ps)}] \hfill \\ Létrehozza a megadott azonosítójú pályát  és feltölti elemekkel. 
					\end{description}
			\end{description}

		\subsubsection{Platform}
				 A Stickman és ezzel a játék mozgásterét korlátozó elem. A framen belül bejárható rész keretezésére szolgál. 			\begin{description}


				\item[Ősosztályok] AbstractFrameItem $\rightarrow{}$ Platform.
				\item[Interfészek] (nincs)
				\item[Attribútumok] (nincs)
				\item[Metódusok] (nincs)
			\end{description}

		\subsubsection{PubSub}
				 Üzenetközvetítő csatorna, mely a Publish/Subscribe mintát valósítja meg. 			\begin{description}


				\item[Ősosztályok] (nincs).
				\item[Interfészek] (nincs)
				\item[Attribútumok]$\ $
					\begin{description}
						\item[\texttt{private Collection<Subscriber> subscribers}] Az eseményekre feliratkozott eseménykezelőket tárolja. 
					\end{description}
				\item[Metódusok]$\ $
					\begin{description}
						\item[\texttt{public void emit(String eventName, Object data)}] \hfill \\ Esemény publikálása 
						\item[\texttt{public void on(String eventName, Subscriber callback)}] \hfill \\ Feliratkozás eseményre 
					\end{description}
			\end{description}

		\subsubsection{Stickman}
				 A játékos által irányított figurát reprezentáló elem. 			\begin{description}


				\item[Ősosztályok] AbstractFrameItem $\rightarrow{}$ Stickman.
				\item[Interfészek] (nincs)
				\item[Attribútumok] (nincs)
				\item[Metódusok]$\ $
					\begin{description}
						\item[\texttt{private Area getNewAreaByDirection(DIRECTION direction)}] \hfill \\ A figura igényelt elmozdulása után elfoglalandó  terület kiszámítása. 
						\item[\texttt{public void move(DIRECTION direction)}] \hfill \\ A figura mozgatása a megadott irányba. 
						\item[\texttt{public void resetToCheckpoint()}] \hfill \\ A figura pozíciójának visszaállítása az  utolsó ellenőrzőpontra. 
					\end{description}
			\end{description}

		\subsubsection{Subscriber} Interfész.
				 A PubSub objektumnak átadható eseménykezelő felülete. 			\begin{description}


				\item[Ősosztályok] (nincs).
				\item[Metódusok]$\ $
					\begin{description}
						\item[\texttt{public void eventEmitted(String eName, Object eParameter)}] \hfill \\ A PubSub objektum által meghívott metódus,  a feliratkozott esemény bekövetkeztekor. 
					\end{description}
			\end{description}

		\subsubsection{Timer}
				 Az idő múlását nyilvántartó objektum.   Elsősorban Stickman esését szabályozhatjuk vele, de a  periodikus időközönként kiadott 'tick' események  a stickmentől vagy bármely más objektumtól függetlenek.  			\begin{description}


				\item[Ősosztályok] (nincs).
				\item[Interfészek] (nincs)
				\item[Attribútumok]$\ $
					\begin{description}
						\item[\texttt{ PubSub pubsub}] Az eseménykezelő csatorna,  melyen jelzi az idő múlását. 
					\end{description}
				\item[Metódusok] (nincs)
			\end{description}

%GENERATOR:CLASS_DESCRIPTIONS

	\subsection{Statikus struktúra diagramok}

		\begin{center}
			\includegraphics[scale=0.7, angle=-90]{resources/model.png}
		\end{center}
			
	\subsection{Szekvencia diagramok}
	
\begin{center}\includegraphics[scale=1]{resources/10LoadGame.png}\end{center}
\begin{center}\includegraphics[scale=0.8]{resources/12AddFrameItemtoMap.png}\end{center}
\begin{center}\includegraphics[scale=1, angle=-90]{resources/11LoadMap.png}\end{center}
\begin{center}\includegraphics[scale=1]{resources/20StickmanMoves.png}\end{center}
\begin{center}\includegraphics[scale=0.8, angle=-90]{resources/21InterframeMove.png}\end{center}
\begin{center}\includegraphics[scale=1]{resources/22FrameChecksCollision.png}\end{center}
\begin{center}\includegraphics[scale=1]{resources/23CheckNeighbouringFrame.png}\end{center}
\begin{center}\includegraphics[scale=1]{resources/24Stickmanfalls.png}\end{center}
\begin{center}\includegraphics[scale=1]{resources/25Pickingupakey.png}\end{center}
\begin{center}\includegraphics[scale=1]{resources/30MoveFrame.png}\end{center}
\begin{center}\includegraphics[scale=0.8, angle=-90]{resources/26Collisionnotification.png}\end{center}

	\subsection{State-chartok}
\begin{center}\includegraphics[scale=1]{resources/10Key.png}\end{center}	
\begin{center}\includegraphics[scale=1]{resources/20Viewport.png}\end{center}

	
	\subsection{Napló}
	% The diary generator uses the following comments to identify the beginning and the ending of the generated diary
	% The following content is auto generated, please do NOT modify, edit the related shared document instead.
	%GENERATOR:DIARY
    \begin{center} 
        \begin{tabular}{| l | p{1.9cm} | p{2.6cm} | p{6.1cm} |}
            \hline
                Kezdet & Időtartam & Résztvevők & Leírás \\
            \hline \hline 
2012. 02. 29. 08:00 & 1.5 óra & Fodor, Thaler & Konzultáció\\ \hline
2012. 02. 29. 16:30 & 1.5 óra & Thaler & Kért módosítások implementálása\\ \hline
2012. 02. 30. 21:00 & 3 óra & Berki & Javítások átnézése\\ \hline
2012. 02. 31. 19:00 & 2 óra & Berki & További javítások\\ \hline
2012. 03. 04. 20:00 & 1.5 óra & Thaler & Sequence diagram javítás\\ \hline
2012. 03. 05. 21:30 & 0.5 óra & Berki & Nyomtatás\\ \hline

            \hline
        \end{tabular}
    \end{center}
%GENERATOR:DIARY
\end{document}
