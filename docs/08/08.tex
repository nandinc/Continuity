\documentclass[12pt,a4paper,oneside]{article}
\usepackage[utf8]{inputenc}
\usepackage{t1enc} % hyphenate accented chars
\usepackage[hungarian]{babel}
\usepackage{../fedlap}
\usepackage{fancyhdr} % elofej, elolab
\usepackage{graphicx}
\usepackage{datetime} % specify date format
\setcounter{secnumdepth}{3} % enable subsubsection

% hasonlitson a doc verziora
\addtolength{\voffset}{-1cm}

% cim
\csapat{nand}{39}
\konzulens{Bozóki Szilárd}
\datum{\todaynum}

% csapattagok
\taga{Berki Endre}{HQNHER}{berkiendre@gmail.com}
\tagb{Fodor Bertalan Ferenc}{H4T1UX}{foberci@gmail.com}
\tagc{Kádár András}{JFENWR}{arycika@gmail.com}
\tagd{Thaler Benedek}{EDDO10}{thalerbenedek@gmail.com}

\setlength{\headheight}{1.3em}
\setlength{\headsep}{2em}

% elofej, elolab
\fancyhf{}

\fancyhead[OL] { \tiny \leftmark{} }
\fancyhead[OR] { \tmpcsapat }

\fancyfoot[OR] { \thepage }
\fancyfoot[OL] { \tmpdatum }

\pagestyle{fancy}

% custom date format, according to customer request
% you have to use the \todaynum command instead of today,
% becouse babel overrides it, and I couldn't find a way to override
% it again. I was tempted to call this format \todaybozoki
\newcommand{\todaynum}{\the\year. \twodigit\month. \twodigit\day}


\usepackage{enumitem}

\begin{document}

\anyag{8. Részletes tervek}
\fedlap

\addtocounter{section}{7}
\section{Szkeleton tervezése}
	\subsection{Osztályok és metódusok tervei}
%GENERATOR:CLASS_DESCRIPTIONS
%GENERATOR:CLASS_DESCRIPTIONS	
	
	\subsection{A tesztek részletes tervei, leírásuk a teszt nyelvén}
			\newcommand{\testitem}[1]{\item \textbf{Név: #1}\\}
			\newcommand{\tdesc}[1]{\textbf{Leírás: } #1\\}
			\newcommand{\tcel}[1]{\textbf{Cél:} #1\\}
	
			\begin{enumerate}[label=\textbf{\arabic*.}, start=1]
%			    \testitem{}
%		        \tdesc{}
%		        \tcel{}
		        
		        \testitem{Pálya betöltése}
		        \tdesc{Betöltésre kerül egy objektumokkal feltöltött pálya, melyet a \texttt{MapFactory} készít el a pálya fájlreprezentációja alapján, és elhelyezi rajta a stickmaneket.}
		        \tcel{Ellenőrizni, hogy a pálya fájlreprezentációjából helyes pályakép generálódik a programban, minden objektum a helyén van-e. \texttt{MapFactory}-nak létre kell hoznia a megfelelő számú \texttt{Frame}-et és azokat helyesen elhelyezni, valamint létre kell hoznia a \texttt{FrameItem}eket.}
		        
		        \testitem{Stickman mozgatása kereten belül}
		        \tdesc{Az egyik stickmant megmozgatjuk kereten belül minden irányba: vízszintesen és horizontálisan (felfele), úgy hogy van tereptárgy a mozgás irányában, mely azt megakadályozza, és úgy is hogy nincs, illetve leugrunk vele egy magaslatról.}
		        \tcel{Ellenőrizni, hogy a kiadott mozgatási parancsok a várt helyre viszik-e a stickmant, a tereptárgyak megakadályozzák-e a mozgását, ill. hogy az ugrás helyesen működik-e (a parancs kiadása után az idő múlása során először fölfele mozog, majd megáll, lefele mozog, és talajt érve újra megáll). A teszt során a \texttt{Stickman} \texttt{area} attribútuma áll középpontban, illetve hogy a \texttt{PubSub}-ból jövő \texttt{tick} események hatására hogyan reagál.}
		        
		        \testitem{Stickman mozgatása keretek között}
		        \tdesc{A Stickmant először olyan keretek között mozgatjuk, melyek átjárhatóak, majd egy olyanba próbáljuk vezérelni, ahova nincs lehetősége átmenni.}
		        \tcel{Keretek közti átjárhatóság megállapítását végző algoritmus, ill. a \texttt{Stickman} egyik \texttt{Frame}-ből másikba való áthelyezésének ellenőrzése. \texttt{Frame} kérésére a \texttt{Map} visszaadja a szomszédos \texttt{Frame}-et, és az előbbi \texttt{Frame} átrakja oda a \texttt{Stickman}t.}
		        
		        \testitem{Stickman kiesése}
		        \tdesc{Stickmant úgy vezéreljük, hogy essen ki egy keret alján úgy, hogy arra nincs több keret, vagy afelé a keret nem átjárható.}
		        \tcel{Ellenőrizzük, hogy a {Stickman} kiesik-e, ha a játék szabályai szerint ki kell esnie, illetve megnézzük, hogy ilyenkor visszakerül-e az utolsó ellenőrzőponthoz. Az aktuális \texttt{Frame} elkéri a \texttt{Map}től a szomszédját, melyre az \texttt{null} értékkel válaszol, erről pedig értesíti a \texttt{Stickmant}.}
		        
		        \testitem{Kulcs felvétele, pálya teljesítése}
		        \tdesc{Stickmannel először megérintjük az ajtót, ezután felvesszük a kucsot, majd megint megérintjük az ajtót.}
		        \tcel{Ellenőrizzük, hogy az ajtó csak kulccsal nyitható-e, illetve hogy a kulcs felvétele megfelelően rögzítődik-e. A teszt középpontjában a \texttt{Frame} által megvalósított ütközésértesítés áll, illetve az állapotok rögzítése, mely a \texttt{PubSub}-on jövő értesítés után a \texttt{Game}-ben történik.}
		        
		        \testitem{Keret mozgatása}
		        \tdesc{Távoli nézetben keret mozgatása parancsot adunk ki.}
		        \tcel{Ellenőrizzük, hogy a keret mozgatása parancs hatására a \texttt{Map} a megfelelő \texttt{Frame}-et mozgatja és azt a megfelelő helyre viszi.}
		        
		        \testitem{Nézetek közötti váltás}
		        \tdesc{A pálya betöltése után a kezdeti távoli nézetből közelibe váltunk. Itt ugrunk egyet, majd megpróbáljuk a kereteket átrendezni, ami nem sikerül. Esünk egyet, mely közben távoli nézetbe váltunk. Ekkor megpróbáljuk vízszintesen mozgatni a stickmant, ami nem sikerül.}
		        \tcel{Ellenőrizzük, hogy a stickman mozgatása le van-e tiltva távoli nézetben, illetve a keretek mozgatása közeli nézetben. Megnézzük még, hogy az időzítés helyesen működik-e, a kiadott \texttt{tick} parancs ellenére sem esik tovább a \texttt{Stickman} távoli nézetben.}
			\end{enumerate}
	
	\subsection{A tesztelést támogató programok tervei}
		A tesztelés során keletkező kimenetek összehasonlítása az előre definiált kimenetekkel nem kézi erővel történik, erre a unix(-like) rendszereken megtalálható \texttt{diff} -- vagy azzal azonos funkcionalitású, más platformon futó -- programot fogunk használni. Egy teszteset akkor tekinthető sikeresnek, ha a \texttt{diff} program nem jelez különbséget a kapott kimenet és az előre definiált kimenet között.


	\subsection{Napló}
	% The diary generator uses the following comments to identify the beginning and the ending of the generated diary
	% The following content is auto generated, please do NOT modify, edit the related shared document instead.
	%GENERATOR:DIARY
   	%GENERATOR:DIARY
\end{document}
