\documentclass[12pt,a4paper,oneside]{article}
\usepackage[utf8]{inputenc}
\usepackage{t1enc} % hyphenate accented chars
\usepackage[hungarian]{babel}
\usepackage{../fedlap}
\usepackage{fancyhdr} % elofej, elolab
\usepackage{graphicx}
\usepackage{datetime} % specify date format
\setcounter{secnumdepth}{3} % enable subsubsection

% hasonlitson a doc verziora
\addtolength{\voffset}{-1cm}

% cim
\csapat{nand}{39}
\konzulens{Bozóki Szilárd}
\datum{\todaynum}

% csapattagok
\taga{Berki Endre}{HQNHER}{berkiendre@gmail.com}
\tagb{Fodor Bertalan Ferenc}{H4T1UX}{foberci@gmail.com}
\tagc{Kádár András}{JFENWR}{arycika@gmail.com}
\tagd{Thaler Benedek}{EDDO10}{thalerbenedek@gmail.com}

\setlength{\headheight}{1.3em}
\setlength{\headsep}{2em}

% elofej, elolab
\fancyhf{}

\fancyhead[OL] { \tiny \leftmark{} }
\fancyhead[OR] { \tmpcsapat }

\fancyfoot[OR] { \thepage }
\fancyfoot[OL] { \tmpdatum }

\pagestyle{fancy}

% custom date format, according to customer request
% you have to use the \todaynum command instead of today,
% becouse babel overrides it, and I couldn't find a way to override
% it again. I was tempted to call this format \todaybozoki
\newcommand{\todaynum}{\the\year. \twodigit\month. \twodigit\day}


\usepackage{enumitem}

\begin{document}

\anyag{5. Szkeleton tervezése}
\fedlap

\addtocounter{section}{4}
\section{Szkeleton tervezése}

	\subsection{A szkeleton modell valóságos use-case-ei}

		\subsubsection{Use-case diagram}
		
		\subsubsection{Use-case leírások}
	
	\subsection{Architektúra}
		Az architektúra felépítésében arra törekedik a csapatunk, hogy a szkeletonban minden egyes use-case letesztelhető legyen és meg lehessen állapítani, hogy az előző, analízisben rögzített szekvencia diagramoknak megfelelően működik minden. A tesztesetek futtatására előre inicializált pályák állnak majd rendelkezésünkre, hogy a fontos szituációk élesben is tesztelhetőek legyenek. A szkeleton tesztelése szekvenciálisan fog lefutni, ezért nincs szükség a konkurrens feladatok vizsgálatára.
		A tesztpályák úgy lettek megválasztva, hogy az adott használati eseteket legjobban mutassák be, minél kevesebb objektummal, hogy a vizsgálandó rész legyen a középpontban. A tesztpályák mellett feltüntettük az általuk használt frame-ek számát, a realizált objektumokat, illetve a tesztelés célját. Ezeken kívül az átláthatóság érdekében az 5.1.2 Use-case leírásokban számozott használati esetek számát, amelyek reprezentálódnak.
								
		\begin{enumerate}[label=\textbf{\arabic*.}, start=0]
		
			\newcommand{\testitem}[1]{\item \textbf{Tesztpálya -- #1}\\}
			\newcommand{\tframe}[1]{\textbf{Frame-ek száma:} #1\\}
			\newcommand{\tobjekt}[1]{\textbf{Objektumok:} #1\\}
			\newcommand{\tcel}[1]{\textbf{Cél:} #1\\}
			\newcommand{\tuse}[1]{\textbf{Tesztelt Use-Case:} #1\\}
		
			\testitem{Inicializálás}
				\tframe{1}
				\tobjekt{Minden, a futtatáshoz szükséges objektumtípusból legalább egy.
					(Game, Timer, PubSub, MapFactory, Map, Frame, Platform, Door, Key, Stickman, stb.)}
				\tcel{Objektumok létrehozása, kapcsolatok kialakításának bemutatása.}
				\tuse{} 
			\testitem{NÉVNÉVNÉV}
				\tframe{}
				\tobjekt{}
				\tcel{} 
				\tuse{}
			\testitem{NÉVNÉVNÉV}
				\tframe{}
				\tobjekt{}
				\tcel{} 
				\tuse{}
			\testitem{NÉVNÉVNÉV}
				\tframe{}
				\tobjekt{}
				\tcel{} 
				\tuse{}
			\testitem{NÉVNÉVNÉV}
				\tframe{}
				\tobjekt{}
				\tcel{} 
				\tuse{}
			\testitem{NÉVNÉVNÉV}
				\tframe{}
				\tobjekt{}
				\tcel{} 
				\tuse{}
			\testitem{NÉVNÉVNÉV}
				\tframe{}
				\tobjekt{}
				\tcel{} 
				\tuse{} 
		\end{enumerate}
	
	\subsection{A szkeleton kezelői felületének terve, dialógusok}
	
	\subsection{Szekvencia diagramok a belső működésre}
	
	\subsection{Napló}
	% The diary generator uses the following comments to identify the beginning and the ending of the generated diary
	% The following content is auto generated, please do NOT modify, edit the related shared document instead.
	%GENERATOR:DIARY
	%GENERATOR:DIARY
\end{document}
