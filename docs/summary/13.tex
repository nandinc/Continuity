\documentclass[12pt,a4paper,oneside]{article}
\usepackage[utf8]{inputenc}
\usepackage{t1enc} % hyphenate accented chars
\usepackage[hungarian]{babel}
\usepackage{../fedlap}
\usepackage{fancyhdr} % elofej, elolab
\usepackage{graphicx}
\usepackage{datetime} % specify date format
\setcounter{secnumdepth}{3} % enable subsubsection

% hasonlitson a doc verziora
\addtolength{\voffset}{-1cm}

% cim
\csapat{nand}{39}
\konzulens{Bozóki Szilárd}
\datum{\todaynum}

% csapattagok
\taga{Berki Endre}{HQNHER}{berkiendre@gmail.com}
\tagb{Fodor Bertalan Ferenc}{H4T1UX}{foberci@gmail.com}
\tagc{Kádár András}{JFENWR}{arycika@gmail.com}
\tagd{Thaler Benedek}{EDDO10}{thalerbenedek@gmail.com}

\setlength{\headheight}{1.3em}
\setlength{\headsep}{2em}

% elofej, elolab
\fancyhf{}

\fancyhead[OL] { \tiny \leftmark{} }
\fancyhead[OR] { \tmpcsapat }

\fancyfoot[OR] { \thepage }
\fancyfoot[OL] { \tmpdatum }

\pagestyle{fancy}

% custom date format, according to customer request
% you have to use the \todaynum command instead of today,
% becouse babel overrides it, and I couldn't find a way to override
% it again. I was tempted to call this format \todaybozoki
\newcommand{\todaynum}{\the\year. \twodigit\month. \twodigit\day}


\usepackage{enumitem}
\usepackage{textcomp}
\usepackage[utf8]{inputenc}
\usepackage[T1]{fontenc}

\begin{document}

\anyag{Összefoglalás}
\fedlap

\addtocounter{section}{13}
\section{Összefoglalás}

	\subsection{A projektre fordított összes munkaidő}
		\begin{center}
		    \begin{tabular}{ | l | l |}
			    \hline
			    \textbf{Név}		& \textbf{Munkaidő}	\\ \hline
				Berki Endre			& 30.5				\\ \hline
			    Fodor Bertalan		& 71.5				\\ \hline
			    Kádár András		& 51				\\ \hline
			    Thaler Benedek		& 97.5				\\ \hline
			    \hline
			    \textbf{Összesen}	& 250.5				\\ \hline
		    \end{tabular}
		\end{center}
	
	\subsection{A feltöltött programok forrássorainak száma}
		\begin{center}
		    \begin{tabular}{ | l | l | l |}
			    \hline
			    \textbf{Fázis}		& \textbf{Kód}	& \textbf{Komment}	\\ \hline
				Szkeleton			& 820			& 666				\\ \hline
			    Protó				& 1527			& 944				\\ \hline
			    Grafikus			& 1794			& 1182				\\ \hline
			    \hline
			    \textbf{Összesen}	& 4141			& 2792				\\ \hline
		    \end{tabular}
		\end{center}
	
	\subsection{Projekt összegzés}
	%A projekt tapasztalatait összegző részben a csapatoknak a projektről kialakult véleményét várjuk. A megválaszolandók köre az alábbi:}
		
		\subsubsection*{Mit tanultak a projektből konkrétan és általában?}
		
		\subsubsection*{Mi volt a legnehezebb és a legkönnyebb?}
		
		\subsubsection*{Összhangban állt-e az idő és a pontszám az elvégzendő feladatokkal?}
		
		\subsubsection*{Ha nem, akkor hol okozott ez nehézséget?}
		
		\subsubsection*{Milyen változtatási javaslatuk van?}
		A program fejlesztése során előfordult, hogy praktikusabbnak láttuk volna a model egy-egy apró részének kisebb változtatását, ami a model számára nagyobb kifejezőerőt, könnyebb használatot biztosított volna. Azonban a dokumentációs rendszert nagyon merevnek találtuk, számolva azzal, hogy mennyi további munkát okozna az összes UML diagram átrajzolása, ellenőrzése, dokumentációba illesztése, valamint a módosított rész összes említésének felkutatása a dokumentációban frissítés céljából -- inkább megmaradtunk az eredeti tervnél. Ez egy igazán nagy project esetében még kifejezetten előnyös is lehet, mivel nincs arra mód, hogy a terveket heti rendszerességgel változtassuk, itt viszont jobban örültünk volna egy valamivel rugalmasabb dokumentációs rendszernek, ami lehetőséget ad a nagyobb fokú adaptivitásra, amellett, hogy friss és részletes képet ad a tervek és a kód aktuális állapotáról.
		
		\subsubsection*{Milyen feladatot ajánlanának a projektre?}
		
	\subsection{Napló}
	% The diary generator uses the following comments to identify the beginning and the ending of the generated diary
	% The following content is auto generated, please do NOT modify, edit the related shared document instead.
	%GENERATOR:DIARY
	%GENERATOR:DIARY
\end{document}
