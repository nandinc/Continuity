\documentclass[12pt,a4paper,oneside]{article}
\usepackage[utf8]{inputenc}
\usepackage{t1enc} % hyphenate accented chars
\usepackage[hungarian]{babel}
\usepackage{../fedlap}
\usepackage{fancyhdr} % elofej, elolab
\usepackage{graphicx}
\usepackage{datetime} % specify date format
\setcounter{secnumdepth}{3} % enable subsubsection

% hasonlitson a doc verziora
\addtolength{\voffset}{-1cm}

% cim
\csapat{nand}{39}
\konzulens{Bozóki Szilárd}
\datum{\todaynum}

% csapattagok
\taga{Berki Endre}{HQNHER}{berkiendre@gmail.com}
\tagb{Fodor Bertalan Ferenc}{H4T1UX}{foberci@gmail.com}
\tagc{Kádár András}{JFENWR}{arycika@gmail.com}
\tagd{Thaler Benedek}{EDDO10}{thalerbenedek@gmail.com}

\setlength{\headheight}{1.3em}
\setlength{\headsep}{2em}

% elofej, elolab
\fancyhf{}

\fancyhead[OL] { \tiny \leftmark{} }
\fancyhead[OR] { \tmpcsapat }

\fancyfoot[OR] { \thepage }
\fancyfoot[OL] { \tmpdatum }

\pagestyle{fancy}

% custom date format, according to customer request
% you have to use the \todaynum command instead of today,
% becouse babel overrides it, and I couldn't find a way to override
% it again. I was tempted to call this format \todaybozoki
\newcommand{\todaynum}{\the\year. \twodigit\month. \twodigit\day}


\usepackage{enumitem}
\usepackage{textcomp}
\usepackage[utf8]{inputenc}
\usepackage[T1]{fontenc}

\begin{document}

\anyag{Összefoglalás}
\fedlap

\addtocounter{section}{13}
\section{Összefoglalás}

	\subsection{A projektre fordított összes munkaidő}
		\begin{center}
		    \begin{tabular}{ | l | l |}
			    \hline
			    \textbf{Név}		& \textbf{Munkaidő}	\\ \hline
				Berki Endre			& 39.5				\\ \hline
			    Fodor Bertalan		& 71.5				\\ \hline
			    Kádár András		& 51				\\ \hline
			    Thaler Benedek		& 97.5				\\ \hline
			    \hline
			    \textbf{Összesen}	& 259.5				\\ \hline
		    \end{tabular}
		\end{center}
	
	\subsection{A feltöltött programok forrássorainak száma}
		\begin{center}
		    \begin{tabular}{ | l | l | l |}
			    \hline
			    \textbf{Fázis}		& \textbf{Kód}	& \textbf{Komment}	\\ \hline
				Szkeleton			& 820			& 666				\\ \hline
			    Protó				& 1527			& 944				\\ \hline
			    Grafikus			& 1794			& 1182				\\ \hline
			    \hline
			    \textbf{Összesen}	& 4141			& 2792				\\ \hline
		    \end{tabular}
		\end{center}
	
	\subsection{Projekt összegzés}
	%A projekt tapasztalatait összegző részben a csapatoknak a projektről kialakult véleményét várjuk. A megválaszolandók köre az alábbi:}
		
		\subsubsection*{Mit tanultak a projektből konkrétan és általában?}
		A projektben a legnagyobb újdonságot és kihívást a csapatmunka rejtette, ennek megfelelően a legfontosabb tanúlságokat is ezekkel kapcsolatban szűrtük le.
		
		Először is el kellett osztani a feladatokat. Úgy kellett részekre bontani a taszkokat, hogy azok minél inkább önmagukban is megálljanak, ne legyen nagy a függés köztük, így pl. ne kelljen sokat a többiekre várakozni. Ez a csapatvezető feladata volt, aki ezt általában a hétvége előtt megtette, így mindenkinek volt elég ideje. Azonban a szükséges idő becslése néha gondot jelentett, így előfordult hogy az egymástól való függés gondot okozott. Megtanultuk hogy az időbeosztásra és a függőségekre kiemelten figyelni kell.
		
		Mivel egy projekten dolgozunk -- még ha próbálkoztunk is -- elkerülhetetlen volt, hogy függés alakuljon ki, ezek feloldása kulcsfontosságú. Nagyon fontos a rendszeres kommunikáció, az információ megosztása, hiszen csak így biztosítható, hogy mindenki birtokában legyen a munkája elvégzéséhez szükséges friss információkkal. Ezt -- a csapat kis méretéből adódóan -- személyesen volt legegyszerűbb megtenni. Ennek megfelelően a projekt elején létrehozott wikit ill. issue-trackert kevéssé használtuk, azonban a projekt vége felé már látszódott, hogy szükséges bizonyos információkat tartósabb módon is rögzíteni. Ez a mi projektméretünknél most még kikerülhető volt a közvetlen kódbeli tárolással, azonban egy kicsit nagyobb rendszernél már elengedhetetlen lenne pl. egy könnyen böngészhető wiki intenzív használata. Az szintén megmutatkozott, hogy a pdf formátumú dokumentáció használhatósága erősen korlátos, nem igazán fordult elő, hogy régebbi beadandókat vettünk volna elő.
		
		\subsubsection*{Mi volt a legnehezebb és a legkönnyebb?}
A projekt során - annak újszerű. az eddigi munkáinktól eltérő megvalósítás-filozófiája miatt - számos ponton ütköztünk kisebb-nagyobb nehézségekbe. A legnagyobb nehézséget talán a modell részleteinek kidolgozása okozta, mely a projekt elkészítési fázisának egyik legkorábbi szakasza volt. A tervezés során nehézséget okozott a modell megfelelő részletességű leírása, az egyes objektumok kisebb részleteit is megfelelően végiggondolni. Eddigi, kisebb komplexitású munkáinknál ez nem bizonyult volna kritikus hibának, itt azonban sok apró hiba is rengeteg pluszmunkával járt.

A modellalkotás nehézségei után a skeleton elkészítése már könnyebb feladat volt, az egyes függvények implementációjára még nem volt szükség, így tulajdonképpen a modell minél nagyobb mértékű ellenőrzése volt a cél, ez az alapos tervezés után a projekt legkönnyebb munkaszakasza volt.
		
		\subsubsection*{Összhangban állt-e az idő és a pontszám az elvégzendő feladatokkal?}
Igen.
		
		\subsubsection*{Milyen változtatási javaslatuk van?}
		A program fejlesztése során előfordult, hogy praktikusabbnak láttuk volna a model egy-egy apró részének kisebb változtatását, ami a model számára nagyobb kifejezőerőt, könnyebb használatot biztosított volna. Azonban a dokumentációs rendszert nagyon merevnek találtuk, számolva azzal, hogy mennyi további munkát okozna az összes UML diagram átrajzolása, ellenőrzése, dokumentációba illesztése, valamint a módosított rész összes említésének felkutatása a dokumentációban frissítés céljából -- inkább megmaradtunk az eredeti tervnél. Ez egy igazán nagy projekt esetében még kifejezetten előnyös is lehet, mivel nincs arra mód, hogy a terveket heti rendszerességgel változtassuk, itt viszont jobban örültünk volna egy valamivel rugalmasabb dokumentációs rendszernek, ami lehetőséget ad a nagyobb fokú adaptivitásra, amellett, hogy friss és részletes képet ad a tervek és a kód aktuális állapotáról.
		
		\subsubsection*{Milyen feladatot ajánlanának a projektre?}
		\begin{description}
\item[Osmos]: A Hemisphere Games által mobilplatformokra fejlesztett játék számítógépes verziójának elkészítése. A játékosnak egy biológiai organizmust kell átvezetnie különböző, eltérő nehézségű pályákon.

\item[Portal 2D]: a Valve nagysikerű játékának egy leegyszerűsített, két dimenziós verziója. A játékosnak adott eszközök segítségével kell logikai rejtvényeket és akadálypályálat megoldania.		
        \end{description}
	\subsection{Napló}
	% The diary generator uses the following comments to identify the beginning and the ending of the generated diary
	% The following content is auto generated, please do NOT modify, edit the related shared document instead.
	%GENERATOR:DIARY
    \begin{center} 
        \begin{tabular}{| l | p{1.9cm} | p{2.6cm} | p{6.1cm} |}
            \hline
                Kezdet & Időtartam & Résztvevők & Leírás \\
            \hline \hline 
2012. 05. 08. 18:00 & 1 óra & Thaler & Összegzés\\ \hline
2012. 05. 08. 20:00 & 1 óra & Kádár & Tex file\\ \hline
2012. 05. 08. 22:00 & 1 óra & Berki & Kiosztott fejezetek\\ \hline
2012. 05. 08. 23:00 & 1 óra & Fodor & Kiosztott fejezet\\ \hline
2012. 05. 08. 24:00 & 0.5 óra & Thaler & Koordináció, Lezárás\\ \hline
2012. 05. 09. 7:30 & 0.5 óra & Berki & Nyomtatás\\ \hline

            \hline
        \end{tabular}
    \end{center}
%GENERATOR:DIARY
\end{document}
