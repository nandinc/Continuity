\documentclass[12pt,a4paper,oneside]{article}
\usepackage[utf8]{inputenc}
\usepackage{t1enc} % hyphenate accented chars
\usepackage[hungarian]{babel}
\usepackage{../fedlap}
\usepackage{fancyhdr} % elofej, elolab
\usepackage{graphicx}
\usepackage{datetime} % specify date format
\setcounter{secnumdepth}{3} % enable subsubsection

% hasonlitson a doc verziora
\addtolength{\voffset}{-1cm}

% cim
\csapat{nand}{39}
\konzulens{Bozóki Szilárd}
\datum{\todaynum}

% csapattagok
\taga{Berki Endre}{HQNHER}{berkiendre@gmail.com}
\tagb{Fodor Bertalan Ferenc}{H4T1UX}{foberci@gmail.com}
\tagc{Kádár András}{JFENWR}{arycika@gmail.com}
\tagd{Thaler Benedek}{EDDO10}{thalerbenedek@gmail.com}

\setlength{\headheight}{1.3em}
\setlength{\headsep}{2em}

% elofej, elolab
\fancyhf{}

\fancyhead[OL] { \tiny \leftmark{} }
\fancyhead[OR] { \tmpcsapat }

\fancyfoot[OR] { \thepage }
\fancyfoot[OL] { \tmpdatum }

\pagestyle{fancy}

% custom date format, according to customer request
% you have to use the \todaynum command instead of today,
% becouse babel overrides it, and I couldn't find a way to override
% it again. I was tempted to call this format \todaybozoki
\newcommand{\todaynum}{\the\year. \twodigit\month. \twodigit\day}


\usepackage{enumitem}
\usepackage{textcomp}
\usepackage[utf8]{inputenc}
\usepackage[T1]{fontenc}

\begin{document}

\anyag{13. Grafikus változat beadása}
\fedlap

\addtocounter{section}{12}
\section{Grafikus változat beadása}

	\subsection{Fordítási és futtatási útmutató}
		%A feltöltött program fordításával és futtatásával kapcsolatos útmutatás. Ennek tartalmaznia kell leltárszer űen az egyes fájlok pontos nevét, méretét byte-ban, keletkezési idejét, valamint azt, hogy a fájlban mi került megvalósításra.
		
		\subsubsection{Fordítás és telepítés}
			%A fenti listában szerepl ő forrásfájlokból milyen műveletekkel lehet a biná ris, futtatható kódot el őállítani, milyen műveletek elvégzése szükséges a telepítéshez. Az előállításhoz csak a 2. Követelmények c. dokumentumban leírt környezetet szabad előírni.
			
		 A fordítás batch fájl segítségével történik. Nyissunk meg egy parancssort, ahol a Java fájlok futtatásához szükséges környezeti változók be vannak állítva (JDK Command Prompt). A könyvtárszerkezet gyökérmappájában állva adjuk ki a \texttt{tools\textbackslash gui\_build.bat} utasítást. Ennek hatására megtörténik a forrásfájlok lefordítása, melyről egy értesítés jelenik meg a képernyőn.

		\subsubsection{Futtatás}
			% A futtatható, telepített rendszer elindításával kapcsolatos teend ők leírása. Az indításhoz csak a 2. Követelmények c. dokumentumban leírt környezetet szabad előírni.	
		
		A fordítás után lehetőségünk nyílik a program futtatására. Ez szintúgy batch fájl segítségével történik a fenti módhoz hasonlóan. \\ A program futtatásához adjuk ki a \texttt{tools\textbackslash gui\_run.bat} utasítást.
		
		\paragraph*{Irányítás -- billentyűk}
		A program alapértelmezetten távoli nézetben indul, melyről a jobb felső sarokban megjelenő ikon tájékoztat. Ebben a nézetben a kurzorok használhatóak a keretek mozgatásához. Közeli nézetbe a szóköz lenyomásával lehet jutni. Ebben a nézetben az első stickman (lány) mozgatására a kurzorok, a második stickman (fiú) mozgatására a \emph{wasd} billentyűk használhatóak.

		\subsubsection{Fájllista}
			\begin{center}
			    \begin{tabular}{ | l | l | l | p{5cm} |}
				    \hline
				    \textbf{Fájl neve}	& \textbf{Méret}	& \textbf{Keltezés ideje}	& \textbf{Tartalom} \\ \hline
				    %Autogenerated content, do not modify
                    %GENERATOR:FILE_LIST
						key\_collected.png		&	194 bytes	&	2012. 05. 06. 22:10:30	&	Megtalált kulcs\\ \hline
						key.png		&	184 bytes	&	2012. 05. 06. 21:12:36	&	Kulcs\\ \hline
						stickman1.png		&	226 bytes	&	2012. 05. 06. 21:12:36	&	Első játékos\\ \hline
						door.png		&	186 bytes	&	2012. 05. 06. 21:12:36	&	Ajtó\\ \hline
						stickman2.png		&	216 bytes	&	2012. 05. 06. 21:12:36	&	Második játékos\\ \hline
						map\_1		&	204 bytes	&	2012. 05. 06. 21:12:36	&	Első pálya\\ \hline
						map\_2		&	394 bytes	&	2012. 05. 06. 21:12:36	&	Második pálya\\ \hline
						map\_3		&	271 bytes	&	2012. 05. 06. 21:12:36	&	Harmadik pálya\\ \hline
						map\_4		&	727 bytes	&	2012. 05. 07. 08:16:52	&	Negyedik pálya\\ \hline
						Application.java		&	817 bytes	&	2012. 05. 06. 11:08:57	&	Application osztály	\\ \hline
						Key.java		&	1690 bytes	&	2012. 04. 07. 23:49:03	&	Key osztály	\\ \hline
						Platform.java		&	958 bytes	&	2012. 04. 01. 22:58:45	&	Platform osztály	\\ \hline
						Timer.java		&	1744 bytes	&	2012. 05. 06. 11:08:57	&	Timer osztály	\\ \hline
						Door.java		&	1207 bytes	&	2012. 04. 15. 23:38:51	&	Door osztály	\\ \hline
						Map.java		&	15508 bytes	&	2012. 05. 04. 22:32:08	&	Map osztály	\\ \hline
						DIRECTION.java		&	332 bytes	&	2012. 04. 01. 20:38:59	&	DIRECTION osztály	\\ \hline
						MapFactory.java		&	5107 bytes	&	2012. 04. 14. 23:31:46	&	MapFactory osztály	\\ \hline
						Frame.java		&	11287 bytes	&	2012. 05. 03. 20:36:05	&	Frame osztály	\\ \hline
			    \end{tabular}
			\end{center}	
			
			%pagebreak					
			
			\begin{center}
			    \begin{tabular}{ | l | l | l | p{5cm} |}
				    \hline
				    \textbf{Fájl neve}	& \textbf{Méret}	& \textbf{Keltezés ideje}	& \textbf{Tartalom} \\ \hline						
						MapErrorException.java		&	562 bytes	&	2012. 04. 01. 23:10:44	&	MapErrorException osztály	\\ \hline
						MapNotFoundException.java		&	225 bytes	&	2012. 04. 01. 23:11:31	&	MapNotFoundException osztály	\\ \hline
						Game.java		&	6783 bytes	&	2012. 05. 06. 21:12:36	&	Game osztály	\\ \hline
						PubSub.java		&	1758 bytes	&	2012. 05. 07. 08:16:52	&	PubSub osztály	\\ \hline
						AbstractFrameItem.java		&	2420 bytes	&	2012. 04. 14. 16:31:47	&	AbstractFrameItem osztály	\\ \hline
						Stickman.java		&	6153 bytes	&	2012. 05. 06. 11:08:57	&	Stickman osztály	\\ \hline
						VIEWPORT\_STATE.java		&	220 bytes	&	2012. 04. 01. 23:21:53	&	VIEWPORT\_STATE osztály	\\ \hline
						Subscriber.java		&	509 bytes	&	2012. 04. 01. 20:38:59	&	Subscriber osztály	\\ \hline
						Area.java		&	4076 bytes	&	2012. 05. 04. 21:48:24	&	Area osztály	\\ \hline
						FrameItem.java		&	1854 bytes	&	2012. 04. 14. 16:31:00	&	FrameItem osztály	\\ \hline
						ItemDrawer.java		&	424 bytes	&	2012. 04. 22. 23:26:45	&	ItemDrawer osztály	\\ \hline
						FrontView.java		&	7058 bytes	&	2012. 05. 07. 08:16:52	&	FrontView osztály	\\ \hline
						DoorDrawer.java		&	998 bytes	&	2012. 05. 07. 08:16:52	&	DoorDrawer osztály	\\ \hline
						KeyDrawer.java		&	1310 bytes	&	2012. 05. 07. 08:16:52	&	KeyDrawer osztály	\\ \hline
						StickmanDrawer.java		&	1335 bytes	&	2012. 05. 07. 08:16:52	&	StickmanDrawer osztály	\\ \hline
						PlatformDrawer.java		&	573 bytes	&	2012. 05. 03. 20:36:05	&	PlatformDrawer osztály	\\ \hline
						FrameDrawer.java		&	1411 bytes	&	2012. 05. 03. 20:36:05	&	FrameDrawer osztály	\\ \hline
						FrontView.java		&	7044 bytes	&	2012. 05. 06. 11:08:57	&	FrontView osztály	\\ \hline
						Logger.java		&	1150 bytes	&	2012. 04. 16. 00:32:01	&	Logger osztály	\\ \hline
						FrontController.java		&	6008 bytes	&	2012. 05. 07. 08:16:52	&	FrontController osztály	\\ \hline
						LoadMap.java		&	325 bytes	&	2012. 04. 14. 16:17:45	&	LoadMap osztály	\\ \hline
						MoveFrame.java		&	369 bytes	&	2012. 04. 14. 16:19:01	&	MoveFrame osztály	\\ \hline
						ViewportSwitch.java		&	296 bytes	&	2012. 04. 14. 16:18:10	&	ViewportSwitch osztály	\\ \hline
						Timer.java		&	574 bytes	&	2012. 04. 14. 17:14:21	&	Timer osztály	\\ \hline
						FrontController.java		&	3572 bytes	&	2012. 04. 16. 01:10:56	&	FrontController osztály	\\ \hline
						Command.java		&	2874 bytes	&	2012. 04. 14. 16:16:19	&	Command osztály	\\ \hline
						InvalidArgumentException.java		&	590 bytes	&	2012. 04. 14. 16:07:57	&	InvalidArgumentException osztály	\\ \hline
						Tick.java		&	554 bytes	&	2012. 04. 14. 16:16:50	&	Tick osztály	\\ \hline
						Move.java		&	435 bytes	&	2012. 04. 14. 16:41:47	&	Move osztály	\\ \hline
						gui\_run.bat		&	58 bytes	&	2012. 05. 06. 22:10:30	&	Program futtatása\\ \hline
						gui\_build.bat		&	487 bytes	&	2012. 05. 07. 08:16:52	&	Program fordítása\\ \hline

					%GENERATOR:FILE_LIST
			    \end{tabular}
			\end{center}

	\subsection{Értékelés}
		\begin{center}
		    \begin{tabular}{ | l | c |}
			    \hline
			    \textbf{Tag neve}	& \textbf{Munka százalékban} 	\\ \hline
				Berki Endre			& 25\%							\\ \hline			    
			    Fodor Bertalan		& 25\%							\\ \hline
			    Kádár András		& 25\%							\\ \hline
			    Thaler Benedek		& 25\%							\\ \hline
		    \end{tabular}
		\end{center}
		
	\subsection{Napló}
	% The diary generator uses the following comments to identify the beginning and the ending of the generated diary
	% The following content is auto generated, please do NOT modify, edit the related shared document instead.
	%GENERATOR:DIARY
    \begin{center} 
        \begin{tabular}{| l | p{1.9cm} | p{2.6cm} | p{6.1cm} |}
            \hline
                Kezdet & Időtartam & Résztvevők & Leírás \\
            \hline \hline 
2012. 04. 27. 8:15 & 0.5 óra & Thaler & Organizáció\\ \hline
2012. 04. 30. 11:30 & 0.5 óra & Kádár & Tex inicializáció\\ \hline
2012. 04. 30. 12:00 & 3 óra & Kádár & Fájllista generátor\\ \hline
2012. 05. 02. 23:00 & 1 óra & Thaler & Support\\ \hline
2012. 05. 03. 16:45 & 0.5 óra & Kádár & Map készítés\\ \hline
2012. 05. 01. 15:00 & 4 óra & Fodor & Grafikus felület kódolása: felépítés, rajzolás\\ \hline
2012. 05. 02. 22:30 & 2 óra & Fodor & Grafikus felület kódolása\\ \hline
2012. 05. 03. 19:00 & 0.5 óra & Fodor & Grafikus felület kódolása\\ \hline
2012. 05. 05. 16:30 & 3 óra & Fodor & Grafikus felület kódolása: kontroller\\ \hline
2012. 05. 05. 14:30 & 2 óra & Thaler & Hibajavítások\\ \hline
2012. 05. 05. 19:30 & 1.5 óra & Fodor & Kódbeli javítások\\ \hline
2012. 05. 06. 19:50 & 0.5 óra & Kádár & Mapok\\ \hline
2012. 05. 06. 15:50 & 3 óra & Berki & Grafikák\\ \hline
2012. 05. 06. 19:20 & 0.5 óra & Fodor & Grafikák beillesztése\\ \hline
2012. 05. 06. 20:30 & 0.5 óra & Fodor & Javítások\\ \hline
2012. 05. 06. 20:45 & 1 óra & Berki & Dokumentáció, tesztelés\\ \hline
2012. 05. 06. 23:10 & 2 óra & Fodor & Javítások\\ \hline
2012. 05. 06. 21:00 & 3 óra & Thaler & Javítás, csomagolás\\ \hline
2012. 05. 06. 8:00 & 1 óra & Thaler & Dokumentáció zárása, csomagolás\\ \hline
2012. 05. 06. 9:30 & 0.5 óra & Berki & Átnézés, nyomtatás\\ \hline

            \hline
        \end{tabular}
    \end{center}
%GENERATOR:DIARY
\end{document}
