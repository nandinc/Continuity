\documentclass[12pt,a4paper,oneside]{article}
\usepackage[utf8]{inputenc}
\usepackage{t1enc} % hyphenate accented chars
\usepackage[hungarian]{babel}
\usepackage{../fedlap}
\usepackage{fancyhdr} % elofej, elolab
\usepackage{graphicx}
\usepackage{datetime} % specify date format
\setcounter{secnumdepth}{3} % enable subsubsection

% hasonlitson a doc verziora
\addtolength{\voffset}{-1cm}

% cim
\csapat{nand}{39}
\konzulens{Bozóki Szilárd}
\datum{\todaynum}

% csapattagok
\taga{Berki Endre}{HQNHER}{berkiendre@gmail.com}
\tagb{Fodor Bertalan Ferenc}{H4T1UX}{foberci@gmail.com}
\tagc{Kádár András}{JFENWR}{arycika@gmail.com}
\tagd{Thaler Benedek}{EDDO10}{thalerbenedek@gmail.com}

\setlength{\headheight}{1.3em}
\setlength{\headsep}{2em}

% elofej, elolab
\fancyhf{}

\fancyhead[OL] { \tiny \leftmark{} }
\fancyhead[OR] { \tmpcsapat }

\fancyfoot[OR] { \thepage }
\fancyfoot[OL] { \tmpdatum }

\pagestyle{fancy}

% custom date format, according to customer request
% you have to use the \todaynum command instead of today,
% becouse babel overrides it, and I couldn't find a way to override
% it again. I was tempted to call this format \todaybozoki
\newcommand{\todaynum}{\the\year. \twodigit\month. \twodigit\day}


\usepackage{enumitem}
\usepackage{textcomp}
\usepackage[utf8]{inputenc}
\usepackage[T1]{fontenc}

\begin{document}

\anyag{13. Grafikus változat beadása}
\fedlap

\addtocounter{section}{12}
\section{Grafikus változat beadása}

	\subsection{Fordítási és futtatási útmutató}
		%A feltöltött program fordításával és futtatásával kapcsolatos útmutatás. Ennek tartalmaznia kell leltárszer űen az egyes fájlok pontos nevét, méretét byte-ban, keletkezési idejét, valamint azt, hogy a fájlban mi került megvalósításra.

		\subsubsection{Fájllista}
			\begin{center}
			    \begin{tabular}{ | l | l | l | p{5cm} |}
				    \hline
				    \textbf{Fájl neve}	& \textbf{Méret}	& \textbf{Keltezés ideje}	& \textbf{Tartalom} \\ \hline
				    %Autogenerated content, do not modify
                    %GENERATOR:FILE_LIST
						Application.java		&	899 bytes	&	2012. 04. 21. 22:03:34	&	Application osztály	\\ \hline
						Command.java		&	2966 bytes	&	2012. 04. 21. 22:03:34	&	Command osztály	\\ \hline
						InvalidArgumentException.java		&	616 bytes	&	2012. 04. 21. 22:03:34	&	InvalidArgumentException osztály	\\ \hline
						FrontController.java		&	3694 bytes	&	2012. 04. 21. 22:03:34	&	FrontController osztály	\\ \hline
						LoadMap.java		&	338 bytes	&	2012. 04. 21. 22:03:34	&	LoadMap osztály	\\ \hline
						Move.java		&	450 bytes	&	2012. 04. 21. 22:03:34	&	Move osztály	\\ \hline
						MoveFrame.java		&	383 bytes	&	2012. 04. 21. 22:03:34	&	MoveFrame osztály	\\ \hline
						Tick.java		&	575 bytes	&	2012. 04. 21. 22:03:34	&	Tick osztály	\\ \hline
						Timer.java		&	592 bytes	&	2012. 04. 21. 22:03:34	&	Timer osztály	\\ \hline
						ViewportSwitch.java		&	308 bytes	&	2012. 04. 21. 22:03:34	&	ViewportSwitch osztály	\\ \hline
						FrontController.java		&	725 bytes	&	2012. 04. 22. 23:48:19	&	FrontController osztály	\\ \hline
						Logger.java		&	1205 bytes	&	2012. 04. 21. 22:03:34	&	Logger osztály	\\ \hline
						AbstractFrameItem.java		&	2523 bytes	&	2012. 04. 21. 22:03:34	&	AbstractFrameItem osztály	\\ \hline
						Area.java		&	4399 bytes	&	2012. 04. 21. 22:03:34	&	Area osztály	\\ \hline
						DIRECTION.java		&	344 bytes	&	2012. 04. 21. 22:03:34	&	DIRECTION osztály	\\ \hline
						Door.java		&	1247 bytes	&	2012. 04. 21. 22:03:34	&	Door osztály	\\ \hline
						MapErrorException.java		&	587 bytes	&	2012. 04. 21. 22:03:34	&	MapErrorException osztály	\\ \hline
						MapNotFoundException.java		&	234 bytes	&	2012. 04. 21. 22:03:34	&	MapNotFoundException osztály	\\ \hline
						Frame.java		&	11661 bytes	&	2012. 04. 21. 22:03:34	&	Frame osztály	\\ \hline
						FrameItem.java		&	1919 bytes	&	2012. 04. 21. 22:03:34	&	FrameItem osztály	\\ \hline
						Game.java		&	6023 bytes	&	2012. 04. 21. 22:03:34	&	Game osztály	\\ \hline
						Key.java		&	1757 bytes	&	2012. 04. 21. 22:03:34	&	Key osztály	\\ \hline
						Map.java		&	15984 bytes	&	2012. 04. 22. 15:23:12	&	Map osztály	\\ \hline
						MapFactory.java		&	5261 bytes	&	2012. 04. 21. 22:03:34	&	MapFactory osztály	\\ \hline
						Platform.java		&	991 bytes	&	2012. 04. 21. 22:03:34	&	Platform osztály	\\ \hline
						PubSub.java		&	1774 bytes	&	2012. 04. 21. 22:03:34	&	PubSub osztály	\\ \hline
						Stickman.java		&	5388 bytes	&	2012. 04. 21. 22:03:34	&	Stickman osztály	\\ \hline
						Subscriber.java		&	525 bytes	&	2012. 04. 21. 22:03:34	&	Subscriber osztály	\\ \hline
						Timer.java		&	2215 bytes	&	2012. 04. 21. 22:03:34	&	Timer osztály	\\ \hline
						VIEWPORT\_STATE.java		&	232 bytes	&	2012. 04. 21. 22:03:34	&	VIEWPORT\_STATE osztály	\\ \hline
						%TODO map_1		&	126 bytes	&	2012. 04. 21. 22:03:34	&	TODO Description.\\ \hline
						%TODO map_2		&	179 bytes	&	2012. 04. 21. 22:03:34	&	TODO Description.\\ \hline
						%TODO map_3		&	112 bytes	&	2012. 04. 21. 22:03:34	&	TODO Description.\\ \hline
						%TODO map_4		&	112 bytes	&	2012. 04. 21. 22:03:34	&	TODO Description.\\ \hline
						%TODO map_5		&	141 bytes	&	2012. 04. 21. 22:03:34	&	TODO Description.\\ \hline
						%TODO map_6		&	222 bytes	&	2012. 04. 21. 22:03:34	&	TODO Description.\\ \hline
						FrontView.java		&	7118 bytes	&	2012. 04. 21. 22:03:34	&	FrontView osztály	\\ \hline
						DoorDrawer.java		&	455 bytes	&	2012. 04. 22. 23:48:19	&	DoorDrawer osztály	\\ \hline
						FrameDrawer.java		&	372 bytes	&	2012. 04. 22. 23:48:19	&	FrameDrawer osztály	\\ \hline
						FrontView.java		&	1406 bytes	&	2012. 04. 22. 23:48:19	&	FrontView osztály	\\ \hline
						ItemDrawer.java		&	441 bytes	&	2012. 04. 22. 23:48:19	&	ItemDrawer osztály	\\ \hline
						KeyDrawer.java		&	454 bytes	&	2012. 04. 22. 23:48:19	&	KeyDrawer osztály	\\ \hline
						PlatformDrawer.java		&	475 bytes	&	2012. 04. 22. 23:48:19	&	PlatformDrawer osztály	\\ \hline
						StickmanDrawer.java		&	474 bytes	&	2012. 04. 22. 23:48:19	&	StickmanDrawer osztály	\\ \hline

					%GENERATOR:FILE_LIST
			    \end{tabular}
			\end{center}

		\subsubsection{Fordítás és telepítés}
			%A fenti listában szerepl ő forrásfájlokból milyen műveletekkel lehet a biná ris, futtatható kódot el őállítani, milyen műveletek elvégzése szükséges a telepítéshez. Az előállításhoz csak a 2. Követelmények c. dokumentumban leírt környezetet szabad előírni.
			
		 A fordítás batch fájl segítségével történik. Nyissunk meg egy parancssort, ahol a Java fájlok futtatásához szükséges környezeti változók be vannak állítva (JDK Command Prompt). A könyvtárszerkezet gyökérmappájában állva adjuk ki a \texttt{tools/gui\_build.bat} utasítást. Ennek hatására megtörténik a forrásfájlok lefordítása, melyről egy értesítés jelenik meg a képernyőn.

		\subsubsection{Futtatás}
			% A futtatható, telepített rendszer elindításával kapcsolatos teend ők leírása. Az indításhoz csak a 2. Követelmények c. dokumentumban leírt környezetet szabad előírni.		
		
		A fordítás után lehetőségünk nyílik a program futtatására. Ez szintúgy batch fájl segítségével történik a fenti módhoz hasonlóan. \\ A program futtatásához adjuk ki a \texttt{tools/gui\_run.bat} utasítást.
%		A tesztelés megkönnyítése érdekében létrehoztunk egy eszközt, mely veszi a tesztesetek bemenetét, ezt beadja a prototípusnak, és a generált kimenetet összehasonlítja az elvárt kimenettel. Ez a program a \texttt{tools/proto\_test.bat} utasítás kiadásával futtatható. A tesztekhez tartozó bemenetek, illetve elvárt kimenetek a \texttt{tools/resources/tests} könyvtárban találhatóak. Fontos, hogy a be- vagy kimenetek eseteleges megváltoztatása esetén a programot újra kell fordítani. 

	\subsection{Értékelés}
		\begin{center}
		    \begin{tabular}{ | l | c |}
			    \hline
			    \textbf{Tag neve}	& \textbf{Munka százalékban} 	\\ \hline
				Berki Endre			& \%							\\ \hline			    
			    Fodor Bertalan		& \%							\\ \hline
			    Kádár András		& \%							\\ \hline
			    Thaler Benedek		& \%							\\ \hline
		    \end{tabular}
		\end{center}
		
	\subsection{Napló}
	% The diary generator uses the following comments to identify the beginning and the ending of the generated diary
	% The following content is auto generated, please do NOT modify, edit the related shared document instead.
	%GENERATOR:DIARY
	%GENERATOR:DIARY
\end{document}
