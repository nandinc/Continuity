\documentclass[12pt,a4paper,oneside]{article}
\usepackage[utf8]{inputenc}
\usepackage{t1enc} % hyphenate accented chars
\usepackage[hungarian]{babel}
\usepackage{../fedlap}
\usepackage{fancyhdr} % elofej, elolab
\usepackage{graphicx}
\usepackage{datetime} % specify date format
\setcounter{secnumdepth}{3} % enable subsubsection

% hasonlitson a doc verziora
\addtolength{\voffset}{-1cm}

% cim
\csapat{nand}{39}
\konzulens{Bozóki Szilárd}
\datum{\todaynum}

% csapattagok
\taga{Berki Endre}{HQNHER}{berkiendre@gmail.com}
\tagb{Fodor Bertalan Ferenc}{H4T1UX}{foberci@gmail.com}
\tagc{Kádár András}{JFENWR}{arycika@gmail.com}
\tagd{Thaler Benedek}{EDDO10}{thalerbenedek@gmail.com}

\setlength{\headheight}{1.3em}
\setlength{\headsep}{2em}

% elofej, elolab
\fancyhf{}

\fancyhead[OL] { \tiny \leftmark{} }
\fancyhead[OR] { \tmpcsapat }

\fancyfoot[OR] { \thepage }
\fancyfoot[OL] { \tmpdatum }

\pagestyle{fancy}

% custom date format, according to customer request
% you have to use the \todaynum command instead of today,
% becouse babel overrides it, and I couldn't find a way to override
% it again. I was tempted to call this format \todaybozoki
\newcommand{\todaynum}{\the\year. \twodigit\month. \twodigit\day}


\usepackage{enumitem}
\usepackage{textcomp}
\usepackage[utf8]{inputenc}
\usepackage[T1]{fontenc}

\begin{document}

\anyag{7. Prototípus koncepciója}
\fedlap

\addtocounter{section}{6}
\section{Prototípus koncepciója}

	\subsection{Prototípus interface-definíciója}
	%Definiálni kell a teszteket leíró nyelvet. Külön figyelmet kell fordítani arra, hogy ha a rendszer véletlen elemeket is tartalmaz, akkor a véletlenszerűség ki-bekapcsolható legyen, és a program determinisztikusan is tesztelhető legyen.
	    \subsubsection{Az interfész általános leírása}
	    %A protó (karakteres) input és output felületeit úgy kell kialakítani, hogy az input fájlból is vehető legyen illetőleg az output fájlba menthető legyen, vagyis kommunikációra csak a szabványos be- és kimenet használható.
	    A prototípus célja, hogy képet kapjunk a szkeletonban külön-külön tesztelt kommunikációs egységek integrálás utáni helyes működéséről. A prototípus egyrészről tehát egy integrációs teszt, másrészről viszont egyszerre egy működőképes program is, mely az egyszerűség érdekében nélkülözi a grafikus felületet, parancssoros kommunikációt használ. A felhasználó a tesztelés során a prototípussal parancsok útján léphet interakcióba, így befolyásolva annak futását. A parancsok futtatása után a kimenet összevethető a specifikációval.
	    
	    \subsubsection{Bemeneti nyelv}
	    %Definiálni kell a teszteket leíró nyelvet. Külön figyelmet kell fordítani arra, hogy ha a rendszer véletlen elemeket is tartalmaz, akkor a véletlenszerűség ki-bekapcsolható legyen, és a program determinisztikusan is futtatható legyen. A szálkezelést is tesztelhető, irányítható módon kell megoldani.
	    %Parancs1 \ Leírás: \ Opciók:
	    
	    A felhasználó a következő parancsok állnak rendelkezésére a tesztelés során:
	    
	    \newcommand{\cmd}[1]{\item{\texttt{#1}} }
	    \begin{description}
	        \cmd{timerStart}: Idő múlását nyilvántartó számláló indítása
	        \cmd{timerStop}: Idő múlását nyilvántartó számláló megállítása
	        \cmd{tick}: Óraütés szimulációja
	        \cmd{loadMap} \emph{mapId}: Megadott pálya betöltése
	        \cmd{move} \emph{stickmanId} \emph{direction}: A megadott azonosítójú stickman megfelelő irányba mozgatása (up, right, down, left)
	        \cmd{viewportSwitch}: Pálya- és Közeli-nézet közötti váltás
	        \cmd{moveFrame} \emph{direction}: Keret mozgatása pályanézetben a megadott irányba (up, right, down, left)
        \end{description}
        
        A parancsokat a program a sztandard bemenetről olvassa be, így tetszőleges fájl beállítható a prancsok forrásának, ha a megfelelő fájlt a bemenetre irányítjuk.
        
        \subsubsection{Pálya fájlreprezentációja}
        %Ha szükséges, meg kell adni a konfigurációs (pl. pályaképet megadó) fájlok nyelvtanát is.
        A felhasznált pályákat a program fájlokból olvassa be, melyek felépítése a következő: A fájlok soralapúak, minden sor egy objektumot kódol a pályán. A sor elején a konkért objektumot azonosító név áll, majd egy tabulátorral elválasztva következnek a paraméterei. A különböző objektumok tetszőleges számú paramétert vehetnek át, mely paramétereket tabulátorral elválasztva kell az objektum neve után felsorolni. Minden, a pályán szereplő objektum első 4 paramétere (ha specifikált), rendre a következő attribútumokat kódolja: Pálya bal felső sarkától számított (abszolút) x irányú eltolás, y irányú eltolás, objektum szélessége, objektum magassága. A pálya fájlban a következő objektumnevek szerepelhetnek: \texttt{Door, Key, Platform, Stickman}. A pályán található keretek számának és elhelyezkedésének megadására nincs szükség, azt a program az elhelyezett elemek pozíciója alapján dinamikusan építi fel.
        
	    \subsubsection{Kimeneti nyelv}	
	    %Egyértelműen definiálni kell, hogy az egyes bemeneti parancsok végrehajtása után előálló állapot milyen formában jelenik meg a szabványos kimeneten.	
	    A karakteres kimeneten minden -- a modell állapotának változását okozó -- esemény után megjelenik a modell aktuális állapotát reprezentáló szöveges ábra. Az ábrán a következő szimbólumok fordulhatnak elő:
	    
	    \begin{description}
	        \newcommand{\frameitem}[1]{\item{\textbf{#1}} }
	        % pdflatex doesn't accept the following symbols, use simpler ones
	        % ☺, ☻, █, ⚷, ☐
	        \frameitem{Ö, Ó}: A stickmanek
	        \frameitem{|, --, +}: Keret függőleges széle, keret vízszintes széle, keret sarka
	        \frameitem{\#}: Platform
	        \frameitem{*}: Kulcs
	        \frameitem{Đ}: Ajtó
        \end{description}

        A pálya szöveges reprezentációja például az alábbi formát öltheti:        
        \begin{verbatim}
+----------+ +----------+ +----------+
|          | |        ##| |#        #|
|  Ö##   ##| |##        | |#   *     |
|##########| |##########| |##########|
+----------+ +----------+ +----------+
+----------+ +----------+
|##   *    | |##     ###|
|    ##    | |    Ó   Đ |
|##########| |##########|
+----------+ +----------+
        \end{verbatim}
	
	\subsection{Összes részletes use-case}
        %domain specific commands
        \newcommand{\ucitem}[1]{\item \textbf{Név: #1}\\}
        \newcommand{\ucdesc}[1]{\textbf{Leírás: } #1\\}
        \newcommand{\ucscenario}[1]{\textbf{Forgatókönyv: }#1\\}
        
        \subsubsection{XYZ Actor use-case-ek} % TODO
		
	    \begin{enumerate}[label=\textbf{\arabic*.}, start=1]
	        \ucitem{UC neve} % TODO
	        \ucdesc{UC leírás} % TODO
	        \ucscenario{UC forgatókönyv} % TODO
	    \end{enumerate}
	
	\subsection{Tesztelési terv}
		\newcommand{\testitem}[1]{\item \textbf{Név: #1}\\}
		\newcommand{\tdesc}[1]{\textbf{Leírás: } #1\\}
		\newcommand{\tcel}[1]{\textbf{Cél:} #1\\}
	
		\begin{enumerate}[label=\textbf{\arabic*.}, start=1]
		    \testitem{T teszt neve} % TODO
	        \tdesc{T teszt leírása} % TODO
	        \tcel{T teszt célja} % TODO
		\end{enumerate}
	
	\subsection{Tesztelést támogató segéd- és fordítóprogramok specifikálása}		


	\subsection{Napló}
	% The diary generator uses the following comments to identify the beginning and the ending of the generated diary
	% The following content is auto generated, please do NOT modify, edit the related shared document instead.
	%GENERATOR:DIARY
    %GENERATOR:DIARY
\end{document}
