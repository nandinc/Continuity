\documentclass[12pt,a4paper,oneside]{article}
\usepackage[utf8]{inputenc}
\usepackage{t1enc} % hyphenate accented chars
\usepackage[hungarian]{babel}
\usepackage{../fedlap}
\usepackage{fancyhdr} % elofej, elolab
\usepackage{graphicx}
\usepackage{datetime} % specify date format
\setcounter{secnumdepth}{3} % enable subsubsection

% hasonlitson a doc verziora
\addtolength{\voffset}{-1cm}

% cim
\csapat{nand}{39}
\konzulens{Bozóki Szilárd}
\datum{\todaynum}

% csapattagok
\taga{Berki Endre}{HQNHER}{berkiendre@gmail.com}
\tagb{Fodor Bertalan Ferenc}{H4T1UX}{foberci@gmail.com}
\tagc{Kádár András}{JFENWR}{arycika@gmail.com}
\tagd{Thaler Benedek}{EDDO10}{thalerbenedek@gmail.com}

\setlength{\headheight}{1.3em}
\setlength{\headsep}{2em}

% elofej, elolab
\fancyhf{}

\fancyhead[OL] { \tiny \leftmark{} }
\fancyhead[OR] { \tmpcsapat }

\fancyfoot[OR] { \thepage }
\fancyfoot[OL] { \tmpdatum }

\pagestyle{fancy}

% custom date format, according to customer request
% you have to use the \todaynum command instead of today,
% becouse babel overrides it, and I couldn't find a way to override
% it again. I was tempted to call this format \todaybozoki
\newcommand{\todaynum}{\the\year. \twodigit\month. \twodigit\day}


\begin{document}

\anyag{6. Szkeleton beadás}
\fedlap

\addtocounter{section}{5}
\section{Szkeleton tervezése}

	\subsection{Fordítási és futtatási útmutató}

		\subsubsection{Fájllista}
			\begin{center}
			    \begin{tabular}{ | l | l | l | p{5cm} |}
				    \hline
				    \textbf{Fájl neve}	& \textbf{Méret}	& \textbf{Keltezés ideje}	& \textbf{Tartalom} \\ \hline
				    %Autogenerated content, do not modify
                    %GENERATOR:FILE_LIST
					AbstractFrameItem.java	& 788 byte	& 2012. 03. 18. 18:33:55	& AbstractFrameItem osztály	\\ \hline
					AbstractTest.java	& 1158 byte	& 2012. 03. 17. 09:11:03	& Szkeleton tesztek ősosztálya	\\ \hline
					AddFrameItem    & 1580 byte	& 2012. 03. 18. 18:14:02	& 1.2 Add FrameItem to Map	\\ %\hline
					ToMapTest.java  &           &                           &                   \\ \hline
					Area.java	& 1832 byte	& 2012. 03. 18. 18:14:02	& Area osztály	\\ \hline
					DIRECTION.java	& 364 byte	& 2012. 03. 17. 08:53:43	& DIRECTION enumeráció	\\ \hline
					Door.java	& 1224 byte	& 2012. 03. 18. 18:14:02	& Door osztály	\\ \hline
					Frame.java	& 6348 byte	& 2012. 03. 18. 18:14:02	& Frame osztály	\\ \hline
					FrameItem.java	& 1229 byte	& 2012. 03. 17. 08:54:13	& FrameItem osztály	\\ \hline
					Game.java	& 1440 byte	& 2012. 03. 18. 18:45:31	& Game osztály	\\ \hline
					Key.java	& 2023 byte	& 2012. 03. 18. 18:33:33	& Key osztály	\\ \hline
					LoadGameTest.java	& 329 byte	& 2012. 03. 18. 18:14:02	& 1.0 Load Game teszt	\\ \hline
					LoadMapTest.java	& 509 byte	& 2012. 03. 18. 18:14:02	& 1.1 Load Map	\\ \hline
					Map.java	& 3180 byte	& 2012. 03. 18. 18:14:02	& Map osztály	\\ \hline
					MapFactory.java	& 1434 byte	& 2012. 03. 18. 18:14:02	& MapFactory osztály	\\ \hline
					PickupKeyTest.java	& 934 byte	& 2012. 03. 18. 18:31:23	& 2.5 Pick up a key	\\ \hline
					Platform.java	& 827 byte	& 2012. 03. 17. 08:55:00	& Platform osztály	\\ \hline
					PubSub.java	& 1762 byte	& 2012. 03. 18. 18:43:20	& PubSub osztály	\\ \hline
					SkeletonLogger.java	& 7961 byte	& 2012. 03. 18. 18:14:02	& Szkeleton kommunikáció követése	\\ \hline
					SkeletonRunner.java	& 4576 byte	& 2012. 03. 18. 18:14:02	& Szkeleton tesztek futtatókörnyezete	\\ \hline
					Stickman.java	& 2690 byte	& 2012. 03. 18. 18:14:02	& Stickman osztály	\\ \hline
					StickmanFalls.java	& 822 byte	& 2012. 03. 18. 18:51:52	& 2.4 Stickman Falls	\\ \hline
					StickmanMovesTest.java	& 618 byte	& 2012. 03. 18. 18:14:02	& 2.0 Stickman Moves	\\ \hline
					Subscriber.java	& 507 byte	& 2012. 03. 17. 08:55:28	& Subscriber interface	\\ \hline
					Test.java	& 241 byte	& 2012. 03. 17. 09:11:47	& Szkeleton tesztek interface	\\ \hline
					Timer.java	& 638 byte	& 2012. 03. 18. 18:14:02	& Timer osztály	\\ \hline

					%GENERATOR:FILE_LIST
			    \end{tabular}
			\end{center}

		\subsubsection{Fordítás}
		 A Szkeleton fordításának menete: a könyvtárszerkezet /skeleton mappájában állva a \\ \texttt{javac utils/*.java} \\ paranccsal lefordul a SkeletonLogger, majd \\ \texttt{javac model/*.java} \\
paranccsal lefordíthatóak a szükséges modellelemek, végül \\ \texttt{javac model/test/*.java} \\utasítással megtörténik a tesztek lefordítása. Ezzel a futtatható fájlok elkészítése megtörtént.

		\subsubsection{Futtatás}
		 A futtatható fájlok elkészítése után \\ \texttt{java model/SkeletonRunner} \\ paranccsal elindítható a teszteket összefogó program, mely a console-on keresztül valósítja meg a felhasználóval történő kommunikációt. \\
 	 A kipróbálható teszteseteket a program kilistázza, mindegyikhez egy számot rendelve, a felhasználó a billentyűzet segítségével adhatja meg a választott tesztesetet. Bizonyos tesztek működése során további interakció szükséges, ilyenkor a program ismét a felhasználó beavatkozására vár. A teszt lefolyásáról a console-on keresztül a felhasználó minden esetben részletes tájékoztatást kap. Egy teszt végrehajtása után a felhasználónak további tesztesetek kipróbálására van lehetősége, a programból való kilépés a '0' bemenenettel hajtható végre.

	\subsection{Értékelés}
		\begin{center}
		    \begin{tabular}{ | l | c |}
			    \hline
			    \textbf{Tag neve}	& \textbf{Munka százalékban} 	\\ \hline
				Berki Endre			& 20.26\%							\\ \hline			    
			    Fodor Bertalan		& 25.72\%							\\ \hline
			    Kádár András		& 22.51\%							\\ \hline
			    Thaler Benedek		& 31.51\%							\\ \hline
		    \end{tabular}
		\end{center}
	\subsection{Napló}
	% The diary generator uses the following comments to identify the beginning and the ending of the generated diary
	% The following content is auto generated, please do NOT modify, edit the related shared document instead.
	%GENERATOR:DIARY
    \begin{center} 
        \begin{tabular}{| l | p{1.9cm} | p{2.6cm} | p{6.1cm} |}
            \hline
                Kezdet & Időtartam & Résztvevők & Leírás \\
            \hline \hline 
2012. 03. 14. 08:00 & 1.5 óra & Fodor, Kádár, Thaler & Konzultáció\\ \hline
2012. 03. 14. 13:30 & 0.5 óra & Kádár & Előkészítés\\ \hline
2012. 03. 14. 15:00 & 0.5 óra & Thaler & Feladatkiosztás\\ \hline
2012. 03. 14. 15:30 & 1 óra & Thaler & Futtatókörnyezet\\ \hline
2012. 03. 14. 18:00 & 1.5 óra & Thaler & Futtatókörnyezet\\ \hline
2012. 03. 14. 20:00 & 0.5 óra & Thaler & Futtatókörnyezet\\ \hline
2012. 03. 16. 15:15 & 1 óra & Thaler & Build környezet\\ \hline
2012. 03. 17. 08:00 & 1.5 óra & Thaler & Teszt demo, dokumentáció\\ \hline
2012. 03. 17. 08:00 & 3 óra & Fodor & SkeletonLogger\\ \hline
2012. 03. 18. 11:00 & 3.5 óra & Kádár & Tesztesetek kódolása\\ \hline
2012. 03. 18. 15:10 & 0.5 óra & Kádár & Tesztesetek kódolása\\ \hline
2012. 03. 18. 16:10 & 1.5 óra & Kádár & Tesztesetek kódolása\\ \hline
2012. 03. 18. 18:00 & 1.5 óra & Thaler & Tesztesetek\\ \hline
2012. 03. 18. 18:00 & 4 óra & Berki & Dokumentáció, futtatókörnyezet\\ \hline
2012. 03. 18. 22:00 & 1 óra & Thaler & Csomagolás\\ \hline
2012. 03. 18. 22:15 & 1 óra & Berki & Átnézés, nyomtatás\\ \hline

            \hline
        \end{tabular}
    \end{center}
%GENERATOR:DIARY
\end{document}