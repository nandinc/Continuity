\documentclass[12pt,a4paper,oneside]{article}
\usepackage[utf8]{inputenc}
\usepackage{t1enc} % hyphenate accented chars
\usepackage[hungarian]{babel}
\usepackage{../fedlap}
\usepackage{fancyhdr} % elofej, elolab
\usepackage{graphicx}
\usepackage{datetime} % specify date format
\setcounter{secnumdepth}{3} % enable subsubsection

% hasonlitson a doc verziora
\addtolength{\voffset}{-1cm}

% cim
\csapat{nand}{39}
\konzulens{Bozóki Szilárd}
\datum{\todaynum}

% csapattagok
\taga{Berki Endre}{HQNHER}{berkiendre@gmail.com}
\tagb{Fodor Bertalan Ferenc}{H4T1UX}{foberci@gmail.com}
\tagc{Kádár András}{JFENWR}{arycika@gmail.com}
\tagd{Thaler Benedek}{EDDO10}{thalerbenedek@gmail.com}

\setlength{\headheight}{1.3em}
\setlength{\headsep}{2em}

% elofej, elolab
\fancyhf{}

\fancyhead[OL] { \tiny \leftmark{} }
\fancyhead[OR] { \tmpcsapat }

\fancyfoot[OR] { \thepage }
\fancyfoot[OL] { \tmpdatum }

\pagestyle{fancy}

% custom date format, according to customer request
% you have to use the \todaynum command instead of today,
% becouse babel overrides it, and I couldn't find a way to override
% it again. I was tempted to call this format \todaybozoki
\newcommand{\todaynum}{\the\year. \twodigit\month. \twodigit\day}


\begin{document}

\anyag{2. Követelmény, projekt, funkcionalitás}
\fedlap
\tableofcontents

\addtocounter{section}{1}
\section{Követelmény, projekt, funkcionalitás}

\subsection{Követelmény definíció}

    \subsubsection{A program bemutatása, alapvető feladatai}
Az elkészített program egy nagymértékben logikai, kismértékben ügyességi játék, mely koncepcióját tekintve teljesen megfelel a Ragtime Games által fejlesztett Continuity játéknak\footnote{http://continuitygame.com/playcontinuity.html}. A játék során a játékos több pályán keresztül tesztelheti logikai és problémamegoldó készségét. Az előre kialakított pályákon az előrehaladás a keretek egymáshoz képest történő elmozdításával, a kereteken belül pedig egy figurával 2 dimenzióban végzett mozgással történik. Minden pálya egy $n*k$-s táblából áll, amin $n*k-1$ keret található. Az összeillő keretek között átjárás van, az üres hellyel szomszédos keretek átmozgathatóak az üres helyre. A játékos által irányított figurának a keretekben található kulcsokat kell összeszednie, hogy azzal a szintén a keretekben található ajtót kinyitva a pályát teljesítse.

    \subsubsection{Felhasználói felület}
A kész program grafikus felhasználói felülettel rendelkezik, melyen az indítással és működéssel kapcsolatos funkciók egy menüsoron keresztül érhetőek el, egér vagy billentyűzet segítségével. Maga a játék kizárólag billentyűzettel irányítható.

    \subsubsection{A program futtatásának követelményei}
A program futtatásához szükséges a futtató számítógépre telepített Java Runtime Environment (JRE), mely ingyenesen beszerezhető a gyártó honlapjáról\footnote{http://java.com/en/download/index.jsp} a legtöbb platformra. A program hardware követelménye megegyezik az Oracle által meghatározott minimum-konfigurációval\footnote{http://www.java.com/en/download/help/sysreq.xml}, mely szükségessé tesz egy kompatibilis operációs rendszert (Windows, Linux), rendszertől függően 64-128 MB memóriát és hozzávetőlegesen 100 MB szabad helyet.

    \subsubsection{A program telepítése}
A program futtatásához a kész terméket csupán a futtató számítógép merevlemezére kell másolni, további telepítés vagy beállítás nem szükséges, amennyiben a futtató számítógép megfelel a futtatás követelményeinek.

    \subsubsection{A program fordítása}
A program fordításához szükség van a \texttt{javac} programra, mellyel a \texttt{*.java} fájlok lefordíthatóak, és a futtatható \texttt{jar} fájl elkészíthető.
    
\subsection{Projekt terv}

    \subsubsection{A felhasznált fejlesztőeszközök}
	A csapat választása az Eclipse\footnote{http://eclipse.org/} fejlesztői rendszerre esett, elsősorban azért, mert mindenkinek tapasztalata van a működésében. Ezen kívül UML modellezéshez, diagramok készítéséhez a Visual Paradigm Community Edition\footnote{http://www.visual-paradigm.com/} kiadását fogjuk használni.
A doukumentáció \LaTeX -ben készül, melyben roppant egyszerű a dokumentálás és a kinézet könnyedén módosítható. A végső fájlokat PDF-be fordítjuk és az alapján nyomtatjuk, amely majd beadásra kerül. A verziókezelést és a felmerülő problémák megoldását a github\footnote{https://github.com/} online és ingyenes felületen keresztül oldjuk meg.

    \subsubsection{A fejlesztőcsapat tagjai, feladatai}
	\begin{center}
	\begin{tabular} {| l | c | }
		\hline
		Név & Feladatok\\
		\hline
		Thaler Benedek & csapatvezetés, dokumentáció, diagramok készítése, kódírás \\ 
		\hline
		Berki Endre & dokumentáció, diagramok készítése, kódírás \\
		\hline
		Fodor Bertalan & dokumentáció, diagramok készítése, kódírás \\
		\hline
		Kádár András & dokumentáció, diagramok készítése, kódírás \\
		\hline
	\end{tabular}
	\end{center}

    \subsubsection{Kommunikációs modell}

    \subsubsection{Fejlesztési mérföldkövek, ütemterv}

	\begin{description}
		\item[Skeleton] \hfill \\
			Az első fontosabb mérföldkő, a szó jelentése: váz, vázrendszer. A program vázának megalkotása. A program modelljének megtervezése, annak helyességének és részletességének ellenőrzése, maga a tervezési folyamat során a gondos és odafigyelő döntéshozás, mert ezek a már elkészült program minőségét nagyban befolyásolják. Ez egy időigényes folyamat, azonban amennyiben sikeresen zárul, a projekt további folyamatai során esetlegesen felmerülő komplikációk száma minimalizálható.
		\item[Prototípus] (a továbbiakban proto)\hfill \\
			A második mérföldkő a program működőképes változata, ami még a grafikát nem tartalmazza. A mérföldkő elérése a program egészének tesztelését teszi lehetővé. A fejlesztés ezen része lezárultnak tekinthető, az algoritmusok végleges formájukba kerülnek, így egy a grafikus felülettül elkülönülő kész programot jelent.
		\item[Grafikus felület] (a továbbiakban GUI)\hfill \\
			A harmadik és egyben utolsó mérföldkövet a grafikus felület befejezése jelenti. Figyelni kell arra, hogy a már kész, jól működő modell köré egy olyan felhasználói környezetet teremtsünk, ami a program használatának hatékonyságát minél jobban maximalizálja. A GUI kialakításánál szem előtt kell tartani a használhatóságot, ergonómiát és a könnyen tanulhatóságot.
	\end{description}

    \subsubsection{Határidők}
	\begin{center}
	\begin{tabular}{| l | l | }
		\hline
		febr. 10. & 14 h -- csapatok regisztrációja\\
		\hline
		febr. 20. & Követelmény, projekt, funkcionalitás -- beadás\\
		\hline
		febr. 27. & Analízis modell kidolgozása 1. -- beadás\\
		\hline
		márc. 5. & Analízis modell kidolgozása 2. -- beadás\\
		\hline
		márc. 12. & Szkeleton tervezése -- beadás\\
		\hline
		márc. 19. & Szkeleton -- beadás\\
		\hline
		márc. 26. & Prototípus koncepciója -- beadás\\
		\hline
		ápr. 2. & Részletes tervek -- beadás\\
		\hline
		ápr. 16. & Prototípus -- beadás\\
		\hline
		ápr. 23. & Grafikus felület specifikációja -- beadás\\
		\hline
		máj. 7. & Grafikus változat -- beadás\\
		\hline
		máj. 11. & Összefoglalás -- beadás\\
		\hline
	\end{tabular}
	\end{center}

    \subsubsection{Átadás}
	A dokumentációt nyomtatott formában minden héten a konzulensnek le kell adni a konzultáció időpontjában. Ezen kívül a három mérföldkő alkalmával a programkódok is bemutatásra kerülnek az ezen célra kijelölt laboratóriumok gépeinél. A végső átadás a félév végeztével az aktualizált dokumentációval és forráskódokkal feltöltésre kerül a tárgy honlapjára\footnote{http://www.iit.bme.hu/hercules}.
%
%   \subsubsection{Kockázatelemzés}
%	\begin{description}
%	\item[Valószínűségek osztályozása:] \hfill \\
%		\begin{description}
%			\item[Alacsony] Közelítőleg 0-20\%-os eséllyel bekövetkező esemény
%			\item[Közepes] Közelítőleg 20-50\%-os eséllyel bekövetkező esemény
%			\item[Biztos] Közelítőleg 50-100\%-os eséllyel bekövetkező esemény
%		\end{description}
%	\item[Hatások osztályozása:] \hfill \\
%		\begin{description}
%			\item[Elhanyagolható] Az esemény bekövetkezése nem okoz különösebb problémát, az esetleges javítása nem igényel komoly munkálatokat.
%			\item[Enyhe] Az esemény bekövetkeztével okozott kár legfeljebb néhány munkaórával javítható, visszaállítható.
%			\item[Közepes] A csapat több tagjának összehangolt munkáját igénylő probléma, melynek megoldása komolyabb erőfeszítéseket igényel. Ezen változtatások már hatással lehetnek a heti ütemtervre, és a fejlesztőcsapat közös megbeszélését igényelhetik.
%			\item[Komoly] A projekt sikeres kimenetelét vagy a csapat integritását fenyegető esemény, mely alapvető változtatásokkal jár. Az ilyen jellegű probléma a csapat azonnali tanácskozását igényli.
%		\end{description}
%	\end{description}
%
%	Eseménytáblázat:
%	\begin{center}
%		\begin{tabular}{| l | l | l |}
%			\hline
%			\textbf{Esemény}			&	\textbf{Valószínűség}	&	\textbf{Hatás}\\
%			\hline
%			Specifikációváltozás			&	Biztos				&	Enyhe\\
%			\hline
%			Javítandó heti beadandó		&	Közepes			&	Enyhe\\
%			\hline
%			Sikertelen heti beadandó		&	Alacsony			&	Közepes\\
%			\hline
%			Csapattag kiválás			&	Alacsony			&	Komoly\\
%			\hline
%			Csapattag betegsége, távolléte	&	Alacsony			&	Közepes\\
%			\hline
%			Hardvermeghibásodás		&	Alacsony			&	Elhanyagolható\\
%			\hline
%			Szoftvermeghibásodás		&	Közepes			&	Enyhe\\
%			\hline
%			Határidorol lecsúszás		&	Alacsony			&	Közepes\\
%			\hline
%			Tanácskozásról hiányzás		&	Közepes			&	Közepes\\
%			\hline
%		\end{tabular}
%	\end{center}
%
   \subsubsection{Egyéb fontos megjegyzések}
	A fejlesztés lefordítása és bemutatása a HSZK laborjaiban történik, így a program forráskódjának kompatibilisnek kell lennie a Java Development Kit ennek megfelelő korábbi verziójával. A maximális kompatibilitás elérése érdekében a fejlesztés is pontosan ezeken a verziószámú platformokon történik majd, azaz ???-es Java SDK és ??? verziójú JRE kerül felhasználásra.

    \subsubsection{Szükséges dokumentációk}
	\begin{enumerate}
	\item Követelmény, projekt, funkcionalitás
	\item Analízis modell kidolgozása 1.
	\item Analízis modell kidolgozása 2.
	\item Szkeleton tervezése
	\item Szkeleton
	\item Prototípus koncepciója
	\item Részletes tervek
	\item Prototípus
	\item Grafikus felület specifikációja
	\item Grafikus változat
	\end{enumerate}
 
\subsection{Feladatleírás}
\subsection{Szótár}

\begin{description}

    \item[Játékos] A programot kezelő felhasználó
    \item[Számítógép] A programot futtató számítógép, mely megfelel a futtatás követelményeinek
    \item[Monitor] A \emph{Számítógéphez} csatlakoztatott képernyő, melyen a futtatott program megjelenik
    \item[Pálya] A játékban való előrehaladás egysége. Minden pálya egy $n*k$ méretű táblázatból áll, melyben $n*k-1$ előre meghatározott \emph{Keret} előre meghatározott helyen található. A fennmaradó helyen egy \emph{Üres keret} található
    \item[Keret] Egy \emph{Pálya} több keretből áll. Minden keret \emph{Elem}eket tartalmaz. Egymáshoz képest \emph{Átrendez}hetőek. Két érintkező keret között egyértelműen megállapítható, hogy \emph{Átjárhatóak}-e. Amennyiben \emph{Stickman} a keret alsó szélét érinti úgy, hogy az \emph{Aktuális keret} nem \emph{Átjárható} az alatta található kerettel, vagy nincs alatta keret, akkor \emph{Stickman} \emph{Kiesik}.
    \item[Átjárható] Két \emph{Keret} átjárható, ha az érintkező oldalon a \emph{Platform}ok minden pontban azonos magasságban vannak.
    \item[Aktuális keret] \emph{Keret}, amiben a játékos lába tartózkodik.
    \item[Üres keret] Az üres keret biztosít lehetőséget arra, hogy vele helyetcserélve a \emph{Keret}ek átrendezhetőek legyenek. Az üres keret nem rendelkezik vizuális reprezentációval, csupán távtartó funkciója van.
    \item[Elem] Egy keret dinamikus vagy statikus alkotórészei; \emph{Stickman, Kulcs, Ajtó, Platform}
    \item[Stickman] A játékban a \emph{Játékos} által irányított emberformájú figura.
    \item[Kulcs] A \emph{Játékos} által \emph{Megszerez}hető \emph{Elem}. A \emph{Pályá}n található összes kulcs megszerzése után van lehetőség az \emph{Ajtó} \emph{Kinyit}ására.
    \item[Ajtó] \emph{Elem}, melyet \emph{Kinyit}va a pálya \emph{Teljesít}ettnek tekinthető.
    \item[Platform] \emph{Elem}, mely meghatározza a \emph{Keret} bejárható és nem bejárható részeit. Egy 2 dimenziós objektum, melyen a \emph{Játékos} irányíthatása szerint \emph{Stickman} mozoghat.
    \item[Megszerez] \emph{Stickman} egy megszerezhető \emph{Elem}et megszerez, ha azt megérinti. Egy elem csak egyszer szerezhető meg, el nem veszíthető.
    \item[Kinyit] \emph{Stickman} az \emph{Ajtó}t kinyithatja, ha \emph{Megszerez}te a \emph{Pályá}n található összes \emph{Kulcs}ot. A kinyitás módja az ajtó megérintése. A kinyitás bekövetkeztekor a \emph{Pálya} \emph{Teljesít}ettnek tekinthető.
    \item[Teljesít (Pályát)] A \emph{Játékos} teljesíti a \emph{Pályát}, ha \emph{Kinyit}ja a pálya \emph{Ajtaját}. Teljesítés után a \emph{Következő pályá}ra léphet.
    \item[Kiesik] Ha \emph{Stickman} kiesik, akkor eltűnik, majd megjelenik az utolsó érintett \emph{Ellenőrzőpont}nál
    \item[Ellenőrzőpont] Ellenőrzőpont a \emph{Pálya} \emph{Kezdőpont}ja, valamint minden \emph{Kulcs} helye.
    \item[Kezdőpont] A \emph{Pálya} egy pontja, ahol \emph{Stickman} először megjelenik a pálya betöltésekor.
    \item[Átrendez] Egy \emph{Keret} átrendezhető, ha mellette \emph{Üres keret} található és \emph{Stickman} csupán egy keretben jelenik meg. Ekkor a két \emph{Keret} helye felcserélhető. Az átrendezést a \emph{Játékos} vezérli.
    \item[Nézet] Meghatározza, hogy a \emph{Játékos} az összes \emph{Keret}et, (\emph{Pálya nézet}) vagy csak az \emph{Aktuális keret}et (\emph{Keret nézet}) lássa.
    \item[Pálya nézet] A \emph{Játékos} az összes \emph{Keret}et látja, és vezérelheti az \emph{Átrendez}ést.
    \item[Keret nézet] A \emph{Játékos} csak az \emph{Aktuális keret}et látja, vezérelheti \emph{Stickman} mozgását.
    \item[Nézetváltás] A \emph{Játékos} utasítására bármikor nézetváltás történhet, mely bekövetkeztekor a grafikus felület ráközelít az \emph{Aktuális keret}re (\emph{Pálya nézet}ről \emph{Keret nézet}re váltás esetén) vagy eltávolodik az a \emph{Aktuális keret}ről az összes keret nézetére (\emph{Pálya nézet})
    \item[Következő pálya] Az aktuális \emph{Pálya} után következő \emph{Pálya} a \emph{Pályalistá}ban. Amennyiben nincs következő pálya a listában, a következő pályának a készítők nevét listázó képernyő tekinthető.
    \item[Pályalista] Az program által tartalmazott összes \emph{Pálya} egy előre definiált sorrendben.

\end{description}

\subsection{Essential use-case-ek}
    %insert figure
	%\begin{figure}[h]
	%	\begin{center}
	%		\includegraphics[scale=0.5]{----FILENAME-HERE----}
	%		\caption{UML diagram test}
	%	\end{center}
	%\end{figure}

\end{document}
