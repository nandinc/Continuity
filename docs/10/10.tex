\documentclass[12pt,a4paper,oneside]{article}
\usepackage[utf8]{inputenc}
\usepackage{t1enc} % hyphenate accented chars
\usepackage[hungarian]{babel}
\usepackage{../fedlap}
\usepackage{fancyhdr} % elofej, elolab
\usepackage{graphicx}
\usepackage{datetime} % specify date format
\setcounter{secnumdepth}{3} % enable subsubsection

% hasonlitson a doc verziora
\addtolength{\voffset}{-1cm}

% cim
\csapat{nand}{39}
\konzulens{Bozóki Szilárd}
\datum{\todaynum}

% csapattagok
\taga{Berki Endre}{HQNHER}{berkiendre@gmail.com}
\tagb{Fodor Bertalan Ferenc}{H4T1UX}{foberci@gmail.com}
\tagc{Kádár András}{JFENWR}{arycika@gmail.com}
\tagd{Thaler Benedek}{EDDO10}{thalerbenedek@gmail.com}

\setlength{\headheight}{1.3em}
\setlength{\headsep}{2em}

% elofej, elolab
\fancyhf{}

\fancyhead[OL] { \tiny \leftmark{} }
\fancyhead[OR] { \tmpcsapat }

\fancyfoot[OR] { \thepage }
\fancyfoot[OL] { \tmpdatum }

\pagestyle{fancy}

% custom date format, according to customer request
% you have to use the \todaynum command instead of today,
% becouse babel overrides it, and I couldn't find a way to override
% it again. I was tempted to call this format \todaybozoki
\newcommand{\todaynum}{\the\year. \twodigit\month. \twodigit\day}


\usepackage{enumitem}
\usepackage{longtable}

\begin{document}

\anyag{10. Prototípus beadása}
\fedlap

\addtocounter{section}{9}
\section{Prototípus beadása}
	\subsection{Fordítási és futtatási útmutató}
%		A feltöltött program fordításával és futtatásával kapcsolatos útmutatás. Ennek tartalmaznia 
%		kell leltárszer űen az egyes fájlok pontos nevét, méretét byte-ban, keletkezési idejét, valamint 
%		azt, hogy a fájlban mi került megvalósításra.

		\subsubsection{Fájllista}
		
			\begin{center} 
				\begin{longtable}{ | l | l | l | p{5cm} |}
					\hline
				    \textbf{Fájl neve}	& \textbf{Méret}	& \textbf{Keltezés ideje}	& \textbf{Tartalom} \\ \hline
				    \hline
				    %Autogenerated content, do not modify
                    %GENERATOR:FILE_LIST
Application.java & 867 byte & 2012-04-14 15:07 & Application osztály \\ \hline
Command.java & 2874 byte & 2012-04-14 16:16 & Command osztály \\ \hline
FrontController.java & 3572 byte & 2012-04-16 01:10 & FrontController osztály \\ \hline
LoadMap.java & 325 byte & 2012-04-14 16:17 & LoadMap osztály \\ \hline
MoveFrame.java & 369 byte & 2012-04-14 16:19 & MoveFrame osztály \\ \hline
Move.java & 435 byte & 2012-04-14 16:41 & Move osztály \\ \hline
Tick.java & 554 byte & 2012-04-14 16:16 & Tick osztály \\ \hline
Timer.java & 574 byte & 2012-04-14 17:14 & Timer osztály \\ \hline
ViewportSwitch.java & 296 byte & 2012-04-14 16:18 & ViewportSwitch osztály \\ \hline
InvalidArgumentException.java & 590 byte & 2012-04-14 16:07 & InvalidArgumentException osztály \\ \hline
Logger.java & 1150 byte & 2012-04-16 00:32 & Logger osztály \\ \hline
AbstractFrameItem.java & 2420 byte & 2012-04-14 16:31 & AbstractFrameItem osztály \\ \hline
Area.java & 4221 byte & 2012-04-07 22:15 & Area osztály \\ \hline
DIRECTION.java & 332 byte & 2012-04-01 20:38 & DIRECTION enumeráció\\ \hline
Door.java & 1207 byte & 2012-04-15 23:38 & Door osztály \\ \hline
FrameItem.java & 1854 byte & 2012-04-14 16:31 & FrameItem osztály \\ \hline
Frame.java & 11284 byte & 2012-04-15 18:33 & Frame osztály \\ \hline
Game.java & 5818 byte & 2012-04-16 00:54 & Game osztály \\ \hline
Key.java & 1690 byte & 2012-04-07 23:49 & Key osztály \\ \hline
MapFactory.java & 5107 byte & 2012-04-14 23:31 & MapFactory osztály \\ \hline
Map.java & 15476 byte & 2012-04-16 00:52 & Map osztály \\ \hline
Platform.java & 958 byte & 2012-04-01 22:58 & Platform osztály \\ \hline
PubSub.java & 1719 byte & 2012-04-16 00:32 & PubSub osztály \\ \hline
Stickman.java & 5189 byte & 2012-04-16 01:06 & Stickman osztály \\ \hline
Subscriber.java & 509 byte & 2012-04-01 20:38 & Subscriber osztály \\ \hline
Timer.java & 2139 byte & 2012-04-14 17:13 & Timer osztály \\ \hline
VIEWPORT\_STATE.java & 220 byte & 2012-04-01 23:21 & VIEWPORT\_STATE enumeráció \\ \hline
MapErrorException.java & 562 byte & 2012-04-01 23:10 & MapErrorException osztály \\ \hline
MapNotFoundException.java & 225 byte & 2012-04-01 23:11 & MapNotFoundException osztály \\ \hline
map\_1 & 120 byte & 2012-04-16 17:40 & 1. térkép fájl \\ \hline
map\_2 & 170 byte & 2012-04-01 23:27 & 2. térkép fájl \\ \hline
map\_3 & 107 byte & 2012-04-02 00:13 & 3. térkép fájl \\ \hline
map\_4 & 107 byte & 2012-04-02 00:13 & 4. térkép fájl \\ \hline
map\_5 & 134 byte & 2012-04-02 00:13 & 5. térkép fájl \\ \hline
map\_6 & 211 byte & 2012-04-15 18:11 & 6. térkép fájl \\ \hline
FrontView.java & 6900 byte & 2012-04-14 16:54 & FrontView osztály \\ \hline
proto\_build.bat & 977 byte & 2012-04-15 22:38 & Indítófájl a prototípus fordításához \\ \hline
proto\_run.bat & 58 byte & 2012-04-15 22:38 & Indítófájl a prototípus futtatásához \\ \hline
proto\_test.bat & 66 byte & 2012-04-15 22:38 & Indítófájl a tesztesetek futtatásához \\ \hline
TestGUI.java & 1805 byte & 2012-04-16 01:09 & TestGUI enumeráció  \\ \hline
TestRunner.java & 2648 byte & 2012-04-16 01:09 & TestRunner osztály \\ \hline
diffutils-1.2.1.jar & 30360 byte & 2012-04-15 22:38 & diff program \\ \hline
1.txt & 10 byte & 2012-04-15 22:38 & 1. teszteset bemenete \\ \hline
2.txt & 127 byte & 2012-04-16 01:03 & 2. teszteset bemenete \\ \hline
3.txt & 203 byte & 2012-04-15 22:38 & 3. teszteset bemenete \\ \hline
4.txt & 38 byte & 2012-04-15 22:38 & 4. teszteset bemenete \\ \hline
5.txt & 73 byte & 2012-04-15 22:38 & 5. teszteset bemenete \\ \hline
6.txt & 40 byte & 2012-04-15 22:38 & 6. teszteset bemenete \\ \hline
7.txt & 79 byte & 2012-04-15 22:38 & 7. teszteset bemenete \\ \hline
1.txt & 709 byte & 2012-04-15 22:38 & 1. teszteset elvárt kimenete \\ \hline
2.txt & 7574 byte & 2012-04-16 01:12 & 2. teszteset elvárt kimenete \\ \hline
3.txt & 11903 byte & 2012-04-15 22:38 & 3. teszteset elvárt kimenete \\ \hline
4.txt & 3677 byte & 2012-04-15 22:38 & 4. teszteset elvárt kimenete \\ \hline
5.txt & 3773 byte & 2012-04-15 22:38 & 5. teszteset elvárt kimenete \\ \hline
6.txt & 1485 byte & 2012-04-15 22:44 & 6. teszteset elvárt kimenete \\ \hline
7.txt & 2334 byte & 2012-04-16 01:12 & 7. teszteset elvárt kimenete \\ \hline
					%GENERATOR:FILE_LIST
				\end{longtable}
			\end{center}
		
		\subsubsection{Fordítás}
%		A fenti listában szerepl ő forrásfájlokból milyen műveletekkel lehet a biná ris, futtatható kódot 
%		el őállítani. Az előállításhoz csak a 2. Követelmények c. dokumentumban leírt környezetet 
%		szabad el őírni.
		
		A fordítás batch fájl segítségével történik. Nyissunk meg egy parancssort, ahol a Java fájlok futtatásához szükséges környezeti változók be vannak állítva (JDK Command Prompt). A könyvtárszerkezet gyökérmappájában állva adjuk ki a \texttt{tools/proto\_build.bat} utasítást. Ennek hatására megtörténik a forrásfájlok lefordítása, melyről egy értesítés jelenik meg a képernyőn.
		
	% AZ ELŐZŐ BEADANDÓBÓL:

%			 A Szkeleton fordításának menete: a könyvtárszerkezet /skeleton mappájában állva a \\ \texttt{javac utils/*.java} \\ paranccsal lefordul a SkeletonLogger, majd \\ \texttt{javac model/*.java} \\ paranccsal lefordíthatóak a szükséges modellelemek, végül \\ \texttt{javac model/test/*.java} \\utasítással megtörténik a tesztek lefordítása. Ezzel a futtatható fájlok elkészítése megtörtént.
		
		\subsubsection{Futtatás}
%		A futtatható kód elindításával kapcsolatos teend ők leírása. Az indításhoz csak a 2. 
%		Követelmények c. dokumentumban leírt környezetet szabad előírni.
		
		A fordítás után lehetőségünk nyílik a prototípus futtatására. Ez szintúgy batch fájl segítségével történik a fenti módhoz hasonlóan.
		
		A prototípus futtatásához adjuk ki a \texttt{tools/proto\_run.bat} utasítást. Ezután lehetőségünk nyílik a korábban definiált parancsok segítségével vezérelni a prototípus futását.
		
		A tesztelés megkönnyítése érdekében létrehoztunk egy eszközt, mely veszi a tesztesetek bemenetét, ezt beadja a prototípusnak, és a generált kimenetet összehasonlítja az elvárt kimenettel. Ez a program a \texttt{tools/proto\_test.bat} utasítás kiadásával futtatható. A tesztekhez tartozó bemenetek, illetve elvárt kimenetek a \texttt{tools/resources/tests} könyvtárban találhatóak. Fontos, hogy a be- vagy kimenetek eseteleges megváltoztatása esetén a programot újra kell fordítani. 
		
	% AZ ELŐZŐ BEADANDÓBÓL:

%			 A futtatható fájlok elkészítése után \\ \texttt{java model/SkeletonRunner} \\ paranccsal elindítható a teszteket összefogó program, mely a console-on keresztül valósítja meg a felhasználóval történő kommunikációt. \\
%	 	 	A kipróbálható teszteseteket a program kilistázza, mindegyikhez egy számot rendelve, a felhasználó a billentyűzet segítségével adhatja meg a választott tesztesetet. Bizonyos tesztek működése során további interakció szükséges, ilyenkor a program ismét a felhasználó beavatkozására vár. A teszt lefolyásáról a console-on keresztül a felhasználó minden esetben részletes tájékoztatást kap. Egy teszt végrehajtása után a felhasználónak további tesztesetek kipróbálására van lehetősége, a programból való kilépés a '0' bemenenettel hajtható végre.
		
	\subsection{Tesztek jegyzőkönyvei}	
	
        \newcommand{\testname}[1]{\hline\textbf{Tesztelő neve} & #1\\ \hline}
        \newcommand{\testtime}[1]{\textbf{Teszt időpontja} & #1\\ \hline}
        \newcommand{\testsumm}[1]{\textbf{Teszt eredménye} & #1\\ \hline}
        \newcommand{\testfault}[1]{\textbf{Lehetséges hibaok} & #1\\ \hline}
        \newcommand{\testmodif}[1]{\textbf{Változtatások} & #1\\ \hline}
		\subsubsection{Pálya betöltése}
%		Az alábbi táblázatot az  utolsó, sikeres tesztfutta táshoz kell kitölteni.
			\begin{center}
				\begin{tabular}{| p{4cm} | p{9cm} |}
					\testname{Fodor Bertalan}
					\testtime{2012. 02. 15. 22:00}
				\end{tabular}
			\end{center}
		
		\subsubsection{Stickman mozgatása kereten belül}
%		Az alábbi táblázatot az  utolsó, sikeres tesztfutta táshoz kell kitölteni.
			\begin{center}
				\begin{tabular}{| p{4cm} | p{9cm} |}
					\testname{Thaler Benedek}
					\testtime{2012. 02. 15. 22:00}
				\end{tabular}
			\end{center}

		\subsubsection{Stickman mozgatása keretek között}
%		Az alábbi táblázatot az  utolsó, sikeres tesztfutta táshoz kell kitölteni.
			\begin{center}
				\begin{tabular}{| p{4cm} | p{9cm} |}
					\testname{Fodor Bertalan}
					\testtime{2012. 02. 15. 23:20}
				\end{tabular}
			\end{center}
			
			\begin{center}
				\begin{tabular}{| p{4cm} | p{9cm} |}
					\testname{Fodor Bertalan}
					\testtime{2012.02.15 22:20}
					\testsumm{Hibás, hiányzó rész a teszt végén.}
					\testfault{Hiányzott egy utasítás a teszt bemenetéből, eggyel kevesebb bemenet volt, mint kimenet.}
					\testmodif{\texttt{> move 1 left} hozzáadása a bemenethez.}
				\end{tabular}
			\end{center}

		\subsubsection{Stickman kiesése}
%		Az alábbi táblázatot az  utolsó, sikeres tesztfutta táshoz kell kitölteni.
			\begin{center}
				\begin{tabular}{| p{4cm} | p{9cm} |}
					\testname{Fodor Bertalan}
					\testtime{2012. 02. 15. 22:40}
				\end{tabular}
			\end{center}

		\subsubsection{Kulcs felvétele, pálya teljesítése}
%		Az alábbi táblázatot az  utolsó, sikeres tesztfutta táshoz kell kitölteni.
			\begin{center}
				\begin{tabular}{| p{4cm} | p{9cm} |}
					\testname{Thaler Benedek}
					\testtime{2012. 02. 15. 22:15}
				\end{tabular}
			\end{center}

		\subsubsection{Keret mozgatása}
%		Az alábbi táblázatot az  utolsó, sikeres tesztfutta táshoz kell kitölteni.
			\begin{center}
				\begin{tabular}{| p{4cm} | p{9cm} |}
					\testname{Fodor Bertalan}
					\testtime{2012. 02. 15. 23:10}
				\end{tabular}
			\end{center}
			
			\begin{center}
				\begin{tabular}{| p{4cm} | p{9cm} |}
					\testname{Fodor Bertalan}
					\testtime{2012.02.15 22:30}
					\testsumm{Hibás, keretmozgatás meghíúsult, ahelyett hogy lefelé mozgott volna.}
					\testfault{Teszt bemenet hibás volt, balra mozgatás szerepelt benne lefele történő mozgatás helyett.}
					\testmodif{Mozgatás irányának módosítása balráról lefelére.}
				\end{tabular}
			\end{center}

		\subsubsection{Nézetek közötti váltás}
%		Az alábbi táblázatot az  utolsó, sikeres tesztfutta táshoz kell kitölteni.
			\begin{center}
				\begin{tabular}{| p{4cm} | p{9cm} |}
					\testname{Thaler Benedek}
					\testtime{2012. 02. 15. 22:50}
				\end{tabular}
			\end{center}

%		Az alábbi táblázatot a megismételt (hibás) tesztek esetén kell kitölteni minden ismétléshez 
%		egyszer. Ha szükséges, akkor a valós kimenet is mellékelhet ő mint a teszt eredménye.
%			\begin{center}
%				\begin{tabular}{| p{4cm} | p{9cm} |}
%					\testname{tesztelő neve}
%					\testtime{időpont}
%					\testsumm{teszt eredménye}
%					\testfault{hiba oka}
%					\testmodif{változtatások}
%				\end{tabular}
%			\end{center}

	\subsection{Értékelés}
%	A projekt kezdete óta az értékelésig eltelt id őben tagokra bontva, százalékban.
		\begin{center}
		    \begin{tabular}{ | l | c |}
			    \hline
			    \textbf{Tag neve}	& \textbf{Munka százalékban} 	\\ \hline
				Berki Endre			& 21\%							\\ \hline			    
			    Fodor Bertalan		& 24\%							\\ \hline
			    Kádár András		& 23\%							\\ \hline
			    Thaler Benedek		& 32\%							\\ \hline
		    \end{tabular}
		\end{center}
	
	\subsection{Napló}
	% The diary generator uses the following comments to identify the beginning and the ending of the generated diary
	% The following content is auto generated, please do NOT modify, edit the related shared document instead.
	%GENERATOR:DIARY
    \begin{center} 
        \begin{tabular}{| l | p{1.9cm} | p{2.6cm} | p{6.1cm} |}
            \hline
                Kezdet & Időtartam & Résztvevők & Leírás \\
            \hline \hline 
2012. 04. 07. 21:00 & 1 óra & Thaler & Prototípus programozás\\ \hline
2012. 04. 12. 14:00 & 1 óra & Kádár & Tex fájl készítése\\ \hline
2012. 04. 12. 15:00 & 1 óra & Kádár & Map generator\\ \hline
2012. 04. 13. 18:30 & 1 óra & Thaler & Prototípus programozás\\ \hline
2012. 04. 13. 21:30 & 0.5 óra & Kádár & Map generator\\ \hline
2012. 04. 14. 12:30 & 4.5 óra & Thaler & Prototípus programozás\\ \hline
2012. 04. 14. 21:00 & 3.5 óra & Kádár & Map generator\\ \hline
2012. 04. 14. 23:30 & 1 óra & Thaler & Map generator\\ \hline
2012. 04. 15. 18:10 & 7 óra & Fodor & Tesztelési eszköz, hibajavítások, dokumentáció\\ \hline
2012. 04. 15. 16:00 & 3 óra & Berki & Batch fájlok, teszt\\ \hline
2012. 04. 15. 19:00 & 1 óra & Berki & Indítási útmutató\\ \hline
2012. 04. 15. 18:30 & 6 óra & Thaler & Takarítás és javítás\\ \hline

            \hline
        \end{tabular}
    \end{center}
%GENERATOR:DIARY
\end{document}
