\documentclass[12pt,a4paper,oneside]{article}
\usepackage[utf8]{inputenc}
\usepackage{t1enc} % hyphenate accented chars
\usepackage[hungarian]{babel}
\usepackage{../fedlap}
\usepackage{fancyhdr} % elofej, elolab
\usepackage{graphicx}
\usepackage{datetime} % specify date format
\setcounter{secnumdepth}{3} % enable subsubsection

% hasonlitson a doc verziora
\addtolength{\voffset}{-1cm}

% cim
\csapat{nand}{39}
\konzulens{Bozóki Szilárd}
\datum{\todaynum}

% csapattagok
\taga{Berki Endre}{HQNHER}{berkiendre@gmail.com}
\tagb{Fodor Bertalan Ferenc}{H4T1UX}{foberci@gmail.com}
\tagc{Kádár András}{JFENWR}{arycika@gmail.com}
\tagd{Thaler Benedek}{EDDO10}{thalerbenedek@gmail.com}

\setlength{\headheight}{1.3em}
\setlength{\headsep}{2em}

% elofej, elolab
\fancyhf{}

\fancyhead[OL] { \tiny \leftmark{} }
\fancyhead[OR] { \tmpcsapat }

\fancyfoot[OR] { \thepage }
\fancyfoot[OL] { \tmpdatum }

\pagestyle{fancy}

% custom date format, according to customer request
% you have to use the \todaynum command instead of today,
% becouse babel overrides it, and I couldn't find a way to override
% it again. I was tempted to call this format \todaybozoki
\newcommand{\todaynum}{\the\year. \twodigit\month. \twodigit\day}


\usepackage{enumitem}

\begin{document}

\anyag{10. Prototípus beadása}
\fedlap

\addtocounter{section}{9}
\section{Prototípus beadása}
	\subsection{Fordítási és futtatási útmutató}
%		A feltöltött program fordításával és futtatásával kapcsolatos útmutatás. Ennek tartalmaznia 
%		kell leltárszer űen az egyes fájlok pontos nevét, méretét byte-ban, keletkezési idejét, valamint 
%		azt, hogy a fájlban mi került megvalósításra.

		\subsubsection{Fájllista}
		
			\begin{center} 
				\begin{tabular}{ | l | l | l | p{5cm} |}
					\hline
				    \textbf{Fájl neve}	& \textbf{Méret}	& \textbf{Keltezés ideje}	& \textbf{Tartalom} \\ \hline
				    \hline
				    %Autogenerated content, do not modify
                    %GENERATOR:FILE_LIST
                    
					%GENERATOR:FILE_LIST
				\end{tabular}
			\end{center}
		
		\subsubsection{Fordítás}
%		A fenti listában szerepl ő forrásfájlokból milyen műveletekkel lehet a biná ris, futtatható kódot 
%		el őállítani. Az előállításhoz csak a 2. Követelmények c. dokumentumban leírt környezetet 
%		szabad el őírni.

	% AZ ELŐZŐ BEADANDÓBÓL:

%			 A Szkeleton fordításának menete: a könyvtárszerkezet /skeleton mappájában állva a \\ \texttt{javac utils/*.java} \\ paranccsal lefordul a SkeletonLogger, majd \\ \texttt{javac model/*.java} \\ paranccsal lefordíthatóak a szükséges modellelemek, végül \\ \texttt{javac model/test/*.java} \\utasítással megtörténik a tesztek lefordítása. Ezzel a futtatható fájlok elkészítése megtörtént.
		
		\subsubsection{Futtatás}
%		A futtatható kód elindításával kapcsolatos teend ők leírása. Az indításhoz csak a 2. 
%		Követelmények c. dokumentumban leírt környezetet szabad előírni.

	% AZ ELŐZŐ BEADANDÓBÓL:

%			 A futtatható fájlok elkészítése után \\ \texttt{java model/SkeletonRunner} \\ paranccsal elindítható a teszteket összefogó program, mely a console-on keresztül valósítja meg a felhasználóval történő kommunikációt. \\
%	 	 	A kipróbálható teszteseteket a program kilistázza, mindegyikhez egy számot rendelve, a felhasználó a billentyűzet segítségével adhatja meg a választott tesztesetet. Bizonyos tesztek működése során további interakció szükséges, ilyenkor a program ismét a felhasználó beavatkozására vár. A teszt lefolyásáról a console-on keresztül a felhasználó minden esetben részletes tájékoztatást kap. Egy teszt végrehajtása után a felhasználónak további tesztesetek kipróbálására van lehetősége, a programból való kilépés a '0' bemenenettel hajtható végre.
		
	\subsection{Tesztek jegyzőkönyvei}
        \newcommand{\testname}[1]{\hline\textbf{Tesztelő neve} & #1\\ \hline}
        \newcommand{\testtime}[1]{\textbf{Teszt időpontja} & #1\\ \hline}
        \newcommand{\testsumm}[1]{\textbf{Teszt eredménye} & #1\\ \hline}
        \newcommand{\testfault}[1]{\textbf{Lehetséges hibaok} & #1\\ \hline}
        \newcommand{\testmodif}[1]{\textbf{Változtatások} & #1\\ \hline}
		\subsubsection{Teszteset 1}
%		Az alábbi táblázatot az  utolsó, sikeres tesztfutta táshoz kell kitölteni.
			\begin{center}
				\begin{tabular}{| p{4cm} | p{9cm} |}
					\testname{tesztelő neve}
					\testtime{időpont}
				\end{tabular}
			\end{center}

%		Az alábbi táblázatot a megismételt (hibás) tesztek esetén kell kitölteni minden ismétléshez 
%		egyszer. Ha szükséges, akkor a valós kimenet is mellékelhet ő mint a teszt eredménye.
			\begin{center}
				\begin{tabular}{| p{4cm} | p{9cm} |}
					\testname{tesztelő neve}
					\testtime{időpont}
					\testsumm{teszt eredménye}
					\testfault{hiba oka}
					\testmodif{változtatások}
				\end{tabular}
			\end{center}
		\subsubsection{Teszteset 2}
		\subsubsection{Teszteset 3}
		\subsubsection{Teszteset 4}
		\subsubsection{Teszteset 5}

	\subsection{Értékelés}
%	A projekt kezdete óta az értékelésig eltelt id őben tagokra bontva, százalékban.
		\begin{center}
		    \begin{tabular}{ | l | c |}
			    \hline
			    \textbf{Tag neve}	& \textbf{Munka százalékban} 	\\ \hline
				Berki Endre			& \%							\\ \hline			    
			    Fodor Bertalan		& \%							\\ \hline
			    Kádár András		& \%							\\ \hline
			    Thaler Benedek		& \%							\\ \hline
		    \end{tabular}
		\end{center}
	
	\subsection{Napló}
	% The diary generator uses the following comments to identify the beginning and the ending of the generated diary
	% The following content is auto generated, please do NOT modify, edit the related shared document instead.
	%GENERATOR:DIARY
	%GENERATOR:DIARY
\end{document}
